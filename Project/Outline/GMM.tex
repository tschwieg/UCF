% Created 2018-07-18 Wed 10:39
\documentclass[11pt]{article}
\usepackage[utf8]{inputenc}
\usepackage[T1]{fontenc}
\usepackage{fixltx2e}
\usepackage{graphicx}
\usepackage{longtable}
\usepackage{float}
\usepackage{wrapfig}
\usepackage{rotating}
\usepackage[normalem]{ulem}
\usepackage{amsmath}
\usepackage{textcomp}
\usepackage{marvosym}
\usepackage{wasysym}
\usepackage{amssymb}
\usepackage{hyperref}
\tolerance=1000
\usepackage{minted}
\DeclareMathOperator*{\argmin}{arg\,min}
\author{Timothy}
\date{\today}
\title{GMM}
\hypersetup{
  pdfkeywords={},
  pdfsubject={},
  pdfcreator={Emacs 25.3.1 (Org mode 8.2.10)}}
\begin{document}

\maketitle
\tableofcontents


\section{Estimation}
\label{sec-1}
It is known that in each time period, the distribution of supply and
demand is binomial. However the difference between two binomial
distributions that are not independent is difficult to estimate using
likelihood methods. As a result, the generalized method of moments
will be utilized. Since the price is uniquely defined in each time
period, as is the quantity supplied, the question of estimation is
feasible.

In each time period T, there exists two moment conditions specified,
one for price, and one for quantity. Under the specification for the
model, for each time period T: $F_V ( p_T^* ) = \prod_{t=1}^T ( 1 - \xi_t )$ and
$q_T^* = \xi_T \prod_{t=1}^T ( 1 - \xi_t )$. This provides us with 2T moment
restrictions on the model, and allows for estimation of up to 2T
parameters. 

A distinction must be made between observations and time
periods. The data are divided into the median price and quantity
sold in each day, and the question of how many data points are in a
time period exists. For the purposes of the estimation in this paper,
I will use 5 observations per time period. If there are N
observations, then there are $T = \ceil{ \frac{ N }{5} }$ time periods.

For the model specified with T time periods, and for a distribution of
prices of log-normal, there are 2 parameters for the distribution, and
T parameters for the $\xi$. There are 2T moment restrictions, so the model
is in fact over-identified. This allows us to test the specification
for our model using the Sargan-Hansen J-test.

\subsection{Complications}
\label{sec-1-1}

One important complication is that there exists a price-floor in the
market. No item is able to be sold at less than \$ 0.03, this means
that for all data points where the price is at this floor, the
equilibrium condition is not binding. Since a price floor leads to
excess supply at the binding price, the only condition that remains
binding is that quantity demanded at the given price is equal to the
quantity sold. Denote K as the number of time periods in which the
price floor is binding.

This condition is written as: $q_d^T = N \prod_{t=1}^T \left [ 1 - \frac{ F_V (
p_T^* ) }{ \prod_{t=1}^{T-1} ( 1 - \xi_t ) } \right ]$. For each time period where
the price is at the floor, there is only one moment condition. For
this model, this implies that there must be at least two time periods
where the price is above the floor in order to identify the
model. This condition is upheld in all the data sets examined in this
paper, and effectively reduces the number of moments. In more
complicated settings with more primitives in the model, this could
become an important problem, as the current specification has the
equilibrium price converging to zero in time.

\subsection{Implementation}
\label{sec-1-2}
Consider a function g(Y$_{\text{t}}$ ,$\mu$,$\sigma$,$\xi$) which gives the moment condition for
each time period, evaluated at the t$^{\text{th}}$ element in that time
period. Under the Null Hypothesis that this model fits the data, then
the expected value of this function is zero.
\begin{equation*}
\mathbb{E}[ g( Y_t, \mu, \sigma, \xi ) ] = 0
\end{equation*}

We seek to estimate the parameters $\mu$, $\sigma$, $\xi$ by minimizing the sample
analog of this with respect to a weighting matrix W. The sample analog
is formed by averaging the data found contained in each time period.
$\hat{m} ( \mu, \sigma, \xi ) = \frac{ 1 }{ M } \sum_{m=1}^M g( Y_m, \mu, \sigma, \xi )$. Let us
combine the parameters of the model into a vector $\theta$. Our goal then
becomes to estimate a value of $\hat{\theta}$ by minimizing the quadratic
form of $\hat(m)$ with respect to matrix W.

\begin{equation*}
\hat{\theta} = \argmin_{\theta} \hat{m}( \theta )' W \hat{m}( \theta )
\end{equation*}

The choice of W is selected by first choosing a positive definite
matrix W, and estimating the model, and then estimating the matrix by
the following method:

\begin{align*}
\hat{W_i} &= \left [ \frac{1}{M} \sum_{m=1}^M g(Y_m, \hat{\theta_{i-1}} ) g( Y_m, \hat{\theta_{i-1}} )' \right ]^{-1} \\
\hat{\theta_i} &= \argmin_{\theta} \hat{m}( \theta_i )' \hat{W_i} \hat{m}( \theta_i ) \\
\end{align*}

This process is then continued until the value of $\theta_{i-1}$ is a
minimizer for W$_{\text{i}}$. This iterated GMM estimator is invariant to the
scale of the data, which is important in this model, as the price and
the quantity data are of wildly different magnitudes. (Cite
Hamilton 1994) This method is also asymptotically equivalent to the
Continuous Updating Efficient GMM, but does not have as many numerical
instabilities.  This process is complicated by $\hat{W_i}$ being of
rank $\min\{ M, 2T - K \}$. If the matrix is not of full rank, then it
is not invertible, and we cannot estimate the model. In order to
ensure that it has full rank, we add a positive number times the
identity matrix to ensure that $\hat{W_i}$ is both positive definite
and invertible.


\subsection{Testing}
\label{sec-1-3}

\subsubsection{Model Fit}
\label{sec-1-3-1}
Since our model is over-identified, we are able to test for model-fit
using the J-test for model fit. Formally, we are testing the
hypothesis that $M \hat{m} ( \hat{\theta} )' \hat{W} \hat{m} ( \hat{\theta}
) = 0$. Since there are 2T - K moments in the
model, and T + 3 primitives in the model, the J-statistic is
distributed $\chi$$_{\text{T-K-3}}^{\text{2}}$. For several of the cases examined, a table
breaking down the model fit is shown.

\begin{center}
\begin{tabular}{lr}
Case & Sargan Test p-Value\\
Glove Case & 1.0\\
Huntsman Case & 0\\
Chroma Case & .89\\
\end{tabular}
\end{center}


\subsubsection{Distribution Rate}
\label{sec-1-3-2}
Of interest is the question of whether or not there has been a
constant drop rate of an item to users in the game over time. This can
be written in the form of: $\textbf{\xi} = \textbf{1} \xi$. That is,
\textbf{\xi} is constant over the entire lifetime of the model. This
hypothesis can be tested with a Likelihood-Ratio test. We estimate the
model under the null and the alternate, and the difference in the
J-stastic is distributed $\chi$$_{\text{T-1}}^{\text{2}}$. 
% Emacs 25.3.1 (Org mode 8.2.10)
\end{document}