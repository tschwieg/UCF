\documentclass[10pt]{paper}\usepackage[]{graphicx}\usepackage[]{color}
%% maxwidth is the original width if it is less than linewidth
%% otherwise use linewidth (to make sure the graphics do not exceed the margin)
\makeatletter
\def\maxwidth{ %
  \ifdim\Gin@nat@width>\linewidth
    \linewidth
  \else
    \Gin@nat@width
  \fi
}
\makeatother

\definecolor{fgcolor}{rgb}{0.345, 0.345, 0.345}
\newcommand{\hlnum}[1]{\textcolor[rgb]{0.686,0.059,0.569}{#1}}%
\newcommand{\hlstr}[1]{\textcolor[rgb]{0.192,0.494,0.8}{#1}}%
\newcommand{\hlcom}[1]{\textcolor[rgb]{0.678,0.584,0.686}{\textit{#1}}}%
\newcommand{\hlopt}[1]{\textcolor[rgb]{0,0,0}{#1}}%
\newcommand{\hlstd}[1]{\textcolor[rgb]{0.345,0.345,0.345}{#1}}%
\newcommand{\hlkwa}[1]{\textcolor[rgb]{0.161,0.373,0.58}{\textbf{#1}}}%
\newcommand{\hlkwb}[1]{\textcolor[rgb]{0.69,0.353,0.396}{#1}}%
\newcommand{\hlkwc}[1]{\textcolor[rgb]{0.333,0.667,0.333}{#1}}%
\newcommand{\hlkwd}[1]{\textcolor[rgb]{0.737,0.353,0.396}{\textbf{#1}}}%
\let\hlipl\hlkwb

\usepackage{framed}
\makeatletter
\newenvironment{kframe}{%
 \def\at@end@of@kframe{}%
 \ifinner\ifhmode%
  \def\at@end@of@kframe{\end{minipage}}%
  \begin{minipage}{\columnwidth}%
 \fi\fi%
 \def\FrameCommand##1{\hskip\@totalleftmargin \hskip-\fboxsep
 \colorbox{shadecolor}{##1}\hskip-\fboxsep
     % There is no \\@totalrightmargin, so:
     \hskip-\linewidth \hskip-\@totalleftmargin \hskip\columnwidth}%
 \MakeFramed {\advance\hsize-\width
   \@totalleftmargin\z@ \linewidth\hsize
   \@setminipage}}%
 {\par\unskip\endMakeFramed%
 \at@end@of@kframe}
\makeatother

\definecolor{shadecolor}{rgb}{.97, .97, .97}
\definecolor{messagecolor}{rgb}{0, 0, 0}
\definecolor{warningcolor}{rgb}{1, 0, 1}
\definecolor{errorcolor}{rgb}{1, 0, 0}
\newenvironment{knitrout}{}{} % an empty environment to be redefined in TeX

\usepackage{alltt}

\usepackage{minted}
\usepackage{amsmath}
\usepackage{float}
\usepackage{amssymb}
\usepackage{geometry}

\title{HW3}
\author{Timothy Schwieg}
\IfFileExists{upquote.sty}{\usepackage{upquote}}{}
\begin{document}
\maketitle

\section*{Question 1}

\subsection*{a}

The Null hypothesis that sons are the same height as their fathers is the joint
hypothesis that $\beta_1 = 0$ and $\beta_2 = 1$. This would imply that $\mathbb{E}[Y_n]
= x_n$ which is the result we are seeking.

\subsection*{b}

\begin{knitrout}
\definecolor{shadecolor}{rgb}{0.969, 0.969, 0.969}\color{fgcolor}\begin{kframe}
\begin{alltt}
\hlstd{galtonData} \hlkwb{<-} \hlkwd{read.table}\hlstd{(} \hlstr{"Galton.dat"}\hlstd{,} \hlkwc{header}\hlstd{=}\hlnum{FALSE} \hlstd{)}
\hlstd{galtonData}\hlopt{$}\hlstd{V2} \hlkwb{<-} \hlstd{galtonData}\hlopt{$}\hlstd{V2} \hlopt{+} \hlnum{60}
\hlstd{galtonData}\hlopt{$}\hlstd{V3} \hlkwb{<-} \hlstd{galtonData}\hlopt{$}\hlstd{V3} \hlopt{+} \hlnum{60}
\hlstd{galtonData}\hlopt{$}\hlstd{V5} \hlkwb{<-} \hlstd{galtonData}\hlopt{$}\hlstd{V5} \hlopt{+} \hlnum{60}
\hlkwd{names}\hlstd{( galtonData )}
\end{alltt}
\begin{verbatim}
## [1] "V1" "V2" "V3" "V4" "V5"
\end{verbatim}
\begin{alltt}
\hlstd{sons} \hlkwb{<-} \hlstd{galtonData[galtonData}\hlopt{$}\hlstd{V4} \hlopt{==} \hlnum{1}\hlstd{,]}

\hlstd{testFStat} \hlkwb{<-} \hlkwa{function}\hlstd{(} \hlkwc{X}\hlstd{,} \hlkwc{Y}\hlstd{,} \hlkwc{R}\hlstd{,} \hlkwc{r} \hlstd{)\{}
    \hlcom{#Calculate The matrix in the most numerically sound method possible}
    \hlstd{Q} \hlkwb{<-} \hlkwd{nrow}\hlstd{( R )}
    \hlcom{#No Numerical problems here officer}
    \hlstd{betaHat} \hlkwb{<-} \hlkwd{solve}\hlstd{(} \hlkwd{t}\hlstd{(X)}\hlopt\hlstd{X)}\hlopt\hlstd{(}\hlkwd{t}\hlstd{(X)}\hlopt\hlstd{Y)}
    \hlstd{E} \hlkwb{<-} \hlstd{Y} \hlopt{-} \hlstd{X}\hlopt\hlstd{betaHat}
    \hlcom{#Seriously why is matrix multiplication so ugly in R}
    \hlstd{sSquared} \hlkwb{<-} \hlstd{(}\hlkwd{t}\hlstd{(E)}\hlopt\hlstd{E)}\hlopt{/}\hlstd{(}\hlkwd{nrow}\hlstd{( sons )} \hlopt{-} \hlkwd{length}\hlstd{(betaHat) )}
    \hlcom{#print( as.numeric(sSquared)*(solve(t(X)%*%X)))}
    \hlcom{#Bang out this bad boy}
    \hlstd{Fstat} \hlkwb{<-} \hlstd{(}\hlkwd{t}\hlstd{(R}\hlopt\hlstd{betaHat} \hlopt{-} \hlstd{r )}\hlopt\hlkwd{solve}\hlstd{(R}\hlopt\hlkwd{solve}\hlstd{(}\hlkwd{t}\hlstd{(X)}\hlopt\hlstd{X)}\hlopt\hlkwd{t}\hlstd{(R))}
        \hlopt\hlstd{(R}\hlopt\hlstd{betaHat} \hlopt{-}\hlstd{r))}\hlopt{/}\hlstd{(Q}\hlopt{*}\hlstd{sSquared)}

    \hlcom{#Finally get what we came for. After all only p-values matter}
    \hlstd{pValue} \hlkwb{<-} \hlkwd{pf}\hlstd{( Fstat,} \hlkwc{lower.tail} \hlstd{=} \hlnum{FALSE}\hlstd{,} \hlkwc{df1} \hlstd{= Q,} \hlkwc{df2} \hlstd{= (}\hlkwd{nrow}\hlstd{(sons)}\hlopt{-}\hlkwd{length}\hlstd{( betaHat)) )}
    \hlkwd{return}\hlstd{(} \hlkwd{c}\hlstd{( pValue, Fstat, betaHat ) )}
\hlstd{\}}
\hlstd{sons}\hlopt{$}\hlstd{V6} \hlkwb{<-} \hlkwd{rep}\hlstd{(} \hlnum{1}\hlstd{,} \hlkwd{nrow}\hlstd{( sons ) )}

\hlstd{X} \hlkwb{<-} \hlkwd{data.matrix}\hlstd{( sons[,}\hlkwd{c}\hlstd{(}\hlnum{6}\hlstd{,}\hlnum{2}\hlstd{)] )}
\hlstd{Y} \hlkwb{<-} \hlkwd{data.matrix}\hlstd{( sons}\hlopt{$}\hlstd{V5 )}

\hlstd{R} \hlkwb{<-} \hlkwd{diag}\hlstd{(}\hlnum{2}\hlstd{)}
\hlstd{r} \hlkwb{<-} \hlkwd{c}\hlstd{(}\hlnum{0}\hlstd{,}\hlnum{1}\hlstd{)}

\hlstd{data} \hlkwb{<-} \hlkwd{testFStat}\hlstd{( X, Y, R, r )}

\hlcom{#P-Value}
\hlkwd{print}\hlstd{( data[}\hlnum{1}\hlstd{] )}
\end{alltt}
\begin{verbatim}
## [1] 6.878526e-27
\end{verbatim}
\begin{alltt}
\hlcom{#F-Stat}
\hlkwd{print}\hlstd{( data[}\hlnum{2}\hlstd{] )}
\end{alltt}
\begin{verbatim}
## [1] 68.49491
\end{verbatim}
\begin{alltt}
\hlcom{#Beta Hat}
\hlkwd{print}\hlstd{( data[}\hlnum{3}\hlopt{:}\hlkwd{length}\hlstd{(data)] )}
\end{alltt}
\begin{verbatim}
## [1] 38.4753827  0.4448583
\end{verbatim}
\begin{alltt}
\hlstd{testReg} \hlkwb{<-}  \hlkwd{lm}\hlstd{( sons}\hlopt{$}\hlstd{V5} \hlopt{~} \hlstd{sons}\hlopt{$}\hlstd{V2 )}
\hlstd{nullReg} \hlkwb{<-}  \hlkwd{lm}\hlstd{( sons}\hlopt{$}\hlstd{V5} \hlopt{~} \hlstd{sons}\hlopt{$}\hlstd{V2} \hlopt{-} \hlnum{1}\hlstd{)}


\hlstd{testStat} \hlkwb{<-} \hlnum{2}\hlopt{*}\hlstd{(} \hlkwd{logLik}\hlstd{( testReg )[}\hlnum{1}\hlstd{]} \hlopt{-} \hlkwd{logLik}\hlstd{( nullReg )[}\hlnum{1}\hlstd{] )}
\hlstd{pValue} \hlkwb{<-} \hlkwd{pchisq}\hlstd{( testStat,} \hlkwc{df} \hlstd{=} \hlnum{1}\hlstd{,} \hlkwc{lower.tail} \hlstd{=} \hlnum{FALSE} \hlstd{)}
\hlstd{pValue}
\end{alltt}
\begin{verbatim}
## [1] 4.158158e-28
\end{verbatim}
\end{kframe}
\end{knitrout}
As we can see, we can reject the null hypothesis at pretty much any significance
level. This leads us to believe that there is some regression to the mean
occurring, and that taller fathers have shorter sons and shorter fathers have
taller sons.

\section*{Question 2}

\subsection*{a}

We may first note that since the arithmatic mean is being used, $x_n =
\frac{x_1+x_2}{2}$ This could also be formatted as $Y_n = \beta_1 + \frac{\beta_2}{2}x_1 + \frac{\beta_2}{2} x_2
+ U_n$. To test differing effects of mother and father heights on the models, we
need to allow those coefficients to vary, so we will consider the slightly
different model of: $Y_n = \beta_1 + \beta_2 x_1 + \beta_3 x_2$. In this framework, the
hypothesis that sons are affected differently by their mother's height than by
their father's height is: $\beta_2 \neq \beta_3$

\subsection*{b}

The Null hypothesis that we wish to test is that: $\beta_2 = \beta_3$.
\begin{knitrout}
\definecolor{shadecolor}{rgb}{0.969, 0.969, 0.969}\color{fgcolor}\begin{kframe}
\begin{alltt}
\hlstd{X} \hlkwb{<-} \hlkwd{data.matrix}\hlstd{( sons[,}\hlkwd{c}\hlstd{(}\hlnum{6}\hlstd{,}\hlnum{2}\hlstd{,}\hlnum{3}\hlstd{)] )}
\hlstd{Y} \hlkwb{<-} \hlkwd{data.matrix}\hlstd{( sons}\hlopt{$}\hlstd{V5 )}

\hlstd{R} \hlkwb{<-} \hlkwd{matrix}\hlstd{(} \hlkwd{c}\hlstd{(}\hlnum{0}\hlstd{,}\hlnum{1}\hlstd{,}\hlopt{-}\hlnum{1}\hlstd{),} \hlkwc{ncol} \hlstd{=} \hlnum{3} \hlstd{)}
\hlstd{r} \hlkwb{<-} \hlkwd{c}\hlstd{(}\hlnum{0}\hlstd{)}

\hlstd{data} \hlkwb{<-} \hlkwd{testFStat}\hlstd{( X, Y, R, r )}

\hlcom{#P-Value}
\hlkwd{print}\hlstd{( data[}\hlnum{1}\hlstd{] )}
\end{alltt}
\begin{verbatim}
## [1] 0.1908053
\end{verbatim}
\begin{alltt}
\hlcom{#F-Stat}
\hlkwd{print}\hlstd{( data[}\hlnum{2}\hlstd{] )}
\end{alltt}
\begin{verbatim}
## [1] 1.716253
\end{verbatim}
\begin{alltt}
\hlcom{#Beta Hat}
\hlkwd{print}\hlstd{( data[}\hlnum{3}\hlopt{:}\hlkwd{length}\hlstd{(data)] )}
\end{alltt}
\begin{verbatim}
## [1] 19.4929408  0.4156405  0.3279953
\end{verbatim}
\begin{alltt}
\hlstd{testReg} \hlkwb{<-}  \hlkwd{lm}\hlstd{( sons}\hlopt{$}\hlstd{V5} \hlopt{~} \hlstd{sons}\hlopt{$}\hlstd{V2} \hlopt{+} \hlstd{sons}\hlopt{$}\hlstd{V3 )}
\hlstd{nullReg} \hlkwb{<-}  \hlkwd{lm}\hlstd{( sons}\hlopt{$}\hlstd{V5} \hlopt{~} \hlkwd{I}\hlstd{(sons}\hlopt{$}\hlstd{V2} \hlopt{+} \hlstd{sons}\hlopt{$}\hlstd{V3))}


\hlstd{testStat} \hlkwb{<-} \hlnum{2}\hlopt{*}\hlstd{(} \hlkwd{logLik}\hlstd{( testReg )[}\hlnum{1}\hlstd{]} \hlopt{-} \hlkwd{logLik}\hlstd{( nullReg )[}\hlnum{1}\hlstd{] )}
\hlstd{pValue} \hlkwb{<-} \hlkwd{pchisq}\hlstd{( testStat,} \hlkwc{df} \hlstd{=} \hlnum{1}\hlstd{,} \hlkwc{lower.tail} \hlstd{=} \hlnum{FALSE} \hlstd{)}
\hlstd{pValue}
\end{alltt}
\begin{verbatim}
## [1] 0.1891877
\end{verbatim}
\end{kframe}
\end{knitrout}
As we can see by the high p-value on both tests, we fail to reject the
null hypothesis unless we are at a very high threshold. We are
left to conclude that it is possible that the height of mother's has the same
effect on the height of a son as the height of father's.
\end{document}
