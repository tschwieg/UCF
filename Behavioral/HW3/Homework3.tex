\documentclass[10pt]{paper}

%\usepackage{minted}
\usepackage{amsmath}
\usepackage{float}
\usepackage{amssymb}
\usepackage{geometry}

\title{Behavioral Homework 3}
\author{Timothy Schwieg}

\begin{document}

\maketitle

\section*{Question 1}
\begin{align*}
  F_V(v) = v \quad \forall v \in [0,1]\\
  f_v(v) = 1 \quad \forall v \in [0,1]\\
\end{align*}

\subsection*{a}
\begin{align*}
  \sigma(v) = v - \frac{\int_0^v w^{N-1} dw}{v^{N-1}}\\
  \sigma(v) = v - \frac{\frac{w^N}{N} \big |_0^v}{v^{N-1}}\\
  \sigma(v) = v - \frac{v^N}{N v^{N-1}} = \frac{(N-1)v}{N}\\
\end{align*}

\subsection*{b}
Since Y is the second order statistic coming from an iid distribution, it is
known that its distribution function is given by:
\begin{align*}
  f_Y = N(N-1)F_V(y)^{N-2}(1-F_v(y))f_v(y)\\
  f_Y = N(N-1)v^{N-2}(1-v)\\
\end{align*}

\subsection*{c}
\begin{align*}
  F_Y(y) = \int_0^yf_Y(v)dv = \int_0^y N(N-1)v^{N-2}(1-v)dv \\
  N(N-1) \big [ \int_0^y v^{N-2}dv - \int_0^y v^{N-1} dv \big ]\\
  N(N-1) \big ( \frac{y^{N-1}}{N-1} - \frac{y^N}{N} \big )\\
  N(N-1) \big ( \frac{y^{N-1}N}{N(N-1)} - \frac{y^N(N-1)}{N(N-1)} \big )\\
  y^{N-1}( N - yN + y)\\
\end{align*}

\subsection*{d}
\begin{align*}
  \mathbb{E}[Y] = \int_0^1 y f_Y(y)dy = \int_0^1 N(N-1)y^{N-1}(1-y)dy\\
  N(N-1) ( \int_0^1 y^{N-1}dy - \int_0^1 y^N dy ) = N(N-1) \big ( \frac{1}{N} -\frac{1}{N+1} \big )\\
  N(N-1) \big ( \frac{N+1}{N(N+1)} - \frac{N}{N(N+1)} \big ) = \frac{N-1}{N+1}\\
\end{align*}

\subsection*{e}

\begin{align*}
  \mathbb{E}[Y^2] = \int_0^1 y^2 f_Y(y)dy = \int_0^1 N(N-1)y^{N}(1-y)dy\\
  N(N-1) ( \int_0^1 y^{N}dy - \int_0^1 y^{N+1} dy ) = N(N-1) \big ( \frac{1}{N+1} -\frac{1}{N+2} \big )\\
  N(N-1) \big ( \frac{N+2}{(N+2)(N+1)} - \frac{N+1}{(N+2)(N+1)} \big ) = \frac{N(N-1)}{(N+1)(N+2)}\\
  \mathbb{V}(Y) = \mathbb{E}[Y^2] -\mathbb{E}[Y]^2\\
  \mathbb{V}(Y) = \frac{N(N-1)}{(N+1)(N+2)} -\frac{(N-1)^2}{(N+1)^2}\\
  \frac{N(N-1)(N+1)}{(N+1)^2(N+2)} - \frac{(N+2)(N-1)^2}{(N+1)^2(N+2)}\\
  \frac{(N-1)^2 }{(N+1)^2(N+2)}\\
\end{align*}

\subsection*{f}

\begin{align*}
  F_Z(z) = F_V(z)^N = z^N \forall v \in [0,1]
\end{align*}

\subsection*{g}

\begin{align*}
  f_Z(z) = \frac{\partial}{\partial z}F_Z(z) = N( z^{N-1} )
\end{align*}

\subsection*{h}

\begin{align*}
  \mathbb{E}[Z] = \int_0^1 zf_Z(z) = N \int_0^1 z^{N} dz = \frac{N}{N+1}
\end{align*}

\subsection*{i}

\begin{align*}
  \mathbb{E}[Z^2] = \int_0^1 N z^{N+1} = \frac{N}{N+2} z^{N+2} \big |_0^1 = \frac{N}{N+2}\\
  Var(Z) = \mathbb{E}[Z^2] - \mathbb{E}[Z]^2 = \frac{N}{N+2}-(\frac{N}{N+1})^2 =\\
  \frac{N(N+1)^2}{(N+2)(N+1)^2} - \frac{N^2(N+2)}{(N+2)(N+1)^2} = \frac{N}{(N+2)(N+1)^2}
\end{align*}

\subsection*{j}
% \sigma(v) = \frac{(N-1)v}{N}
First we must verify that $\sigma (v)$ is a monotonic transformation.
\begin{align*}
  \frac{\partial}{\partial v}\sigma (v) = \frac{N-1}{N} > 0
\end{align*}
Since it is a monotonic transformation, as the probability that the first order
statistic is zero is 0 almost surely, we may proceed with the method of
transformations for finding the pdf of W.
\begin{align*}
  \sigma^{-1}(w) = \frac{N}{N-1} w\\
  \frac{\partial \sigma^{-1}}{\partial w} = \frac{N}{N-1}\\
  f_W = f_{V_{(1:N)}}( \sigma^{-1}(w ) ) \big | \frac{\partial \sigma^{-1}}{\partial w} \big |\\
  f_W = N( \frac{N-1}{N} w )^{N-1} \frac{ N}{N-1}\\
\end{align*}

\subsection*{k}
\begin{align*}
  \mathbb{E}[W] = \mathbb{E}[\frac{(N-1)v}{N}] = \frac{N-1}{N} \mathbb{E}[V_{(1:N)}]\\
  \frac{N-1}{N} \frac{N}{N+1} = \frac{N-1}{N+1}
\end{align*}

\subsection*{l}

\begin{align*}
  \mathbb{V}(W) = \frac{(N-1)^2}{N^2} \mathbb{V}(V_{(1:N)}) = \\
  \frac{(N-1)^2}{N^2} \frac{N}{(N+2)(N+1)^2} = \frac{(N-1)^2}{N(N+2)(N+1)}
\end{align*}


\section*{Question 2}

\subsection*{a}

Logically, the bidder will wish to not place any bid if his valuation is below
the reserve price, so we shall assume that the new bids are uniform on the interval
[r,1] and there will be M = $\sum_{n=1}^N 1_{v_n \geq r}$ participants. This  will be
considered this exogenous.

\begin{align*}
  \max \mathbb{E}[ (v-s)P( win | s ) ] \\
  \max (v-s)F_v(\sigma^{-1}(s_m))^{M-1}\\
  \max (v-s) (\frac{\sigma^{-1}(s_m) -r }{1-r})^{M-1}\\
  (v-s)(M-1)(\frac{\sigma^{-1}(s_m) - r}{1-r})^{M-2} \frac{1}{1-r} \frac{1}{\sigma'(v)} - (\frac{\sigma^{-1}(s_m) -r}{1-r})^{M-1} = 0\\
  (v-\sigma(v))(M-1)(\frac{v - r}{1-r})^{M-2} (\frac{1}{1-r}) \frac{1}{\sigma'(v)} - (\frac{v -r}{1-r})^{M-1} = 0\\
  \sigma'(v) (\frac{v-r}{1-r})^{M-1} + \sigma(v) (\frac{v-r}{1-r})^{M-2} (M-1) \frac{1}{1-r} = v(M-1)(\frac{v-r}{1-r})^{M-2} \frac{1}{1-r}\\
  \sigma'(v) + \sigma(v) \frac{M-1}{v-r} = \frac{v(M-1)}{v-r} \text{ Applying } \mu = \exp( \int \frac{M-1}{v-r}) = (v-r)^{M-1}\\
  (\sigma(v)(v-r)^{M-1})' = v(M-1)(v-r)^{M-2}\\
  \sigma(v)(v-r)^{M-1} = v(v-r)^{M-1} - \frac{(v-r)^M}{M} + C\\
  \sigma(v) = \frac{M-1}{M}v + \frac{r}{M} + C(v-r)^{1-M}\\
  \sigma(v) = \frac{M-1}{M}v + \frac{r}{M}\\
\end{align*}

\subsection*{b}

Since M = $\sum_{n=1}^N 1_{v_n \geq r}$ we can see that it is the sum of
bernoulli random variables and thus is binomial. The probability that each event
occurs is $P(v_n \geq r ) = 1 - F_V(r ).$ Thus $M \sim binom( N, 1- F_V(r) )$.

\subsection*{c}
The optimal reserve price can be found by solving the equation
\begin{align*}
  r^* = v^0 - \frac{1-F_V(r^*)}{f_V(r^*)}\\
  r^* =v^0 + \frac{1-r^*}{1}\\
  2r^* = v^0 + 1\\
  r^* = \frac{v^0 + 1}{2}\\
\end{align*}

\end{document}