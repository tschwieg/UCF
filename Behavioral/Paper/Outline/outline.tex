% Created 2018-03-23 Fri 18:35
\documentclass[11pt]{article}
\usepackage[utf8]{inputenc}
\usepackage[T1]{fontenc}
\usepackage{fixltx2e}
\usepackage{graphicx}
\usepackage{longtable}
\usepackage{float}
\usepackage{wrapfig}
\usepackage{rotating}
\usepackage[normalem]{ulem}
\usepackage{amsmath}
\usepackage{textcomp}
\usepackage{marvosym}
\usepackage{wasysym}
\usepackage{amssymb}
\usepackage{hyperref}
\tolerance=1000
\usepackage{natbib}
\author{Timothy Schwieg}
\date{\today}
\title{Outline}
\hypersetup{
  pdfkeywords={},
  pdfsubject={},
  pdfcreator={Emacs 25.3.1 (Org mode 8.2.10)}}
\begin{document}

\maketitle
\bibliographystyle{chicago}



\section{Reading list}
\label{sec-1}
\nocite{Efficiency}
\nocite{DoubleAuc}
\nocite{LimeBoy}
\nocite{LitReview}

\bibliography{biblio}

\section{Model}
\label{sec-2}
\subsection{General Approach}
\label{sec-2-1}

Because of the nature of the data, as well as the complicated
structure that Cumulative Prospect theory places upon the valuation
function, it is difficult to use it within the structure used commonly
in double auction literature. Almost all models of auctions specify a
structure between the valuation and the price that takes the form of
the valuation is the valuation for the item minus the price paid for
the item at the auction. This difference cannot be reconciled with the
framing effects required by Cumulative Prospect Theory.

To circumvent this issue, I use the result found in \cite{Efficiency}
which states that a continuous double auction converges quickly, in
order O(N$^{\text{2}}$) to a competitive market. From this, I can consider that in
large double auctions, such as is the case for the purchases of the
lotteries in question. From this we can conduct our analysis as if a
buyer of a lottery is at a competitive market, sees the price given by
the market and makes a decision choice based upon his valuation, as a
function of the price. 

This leads us to one of two outcomes in the market, either there is a
purchase made, indicating a valuation above the price seen by the
consumer, or he leaves a buy order, indicating a price he would be
willing to pay. Since in a double auction where the buyer pays the
seller's price, it is dominant to tell the truth I will consider these
buy orders as the actual valuations of the consumer \cite{PriceDataOnly}. 


\subsection{Discrete Choice}
\label{sec-2-2}
By Following the discrete choice model, we assert that decisions made
by the individual are made such that if the valuations of lottery plus
$\xi_l$ is greater than the valuation of not purchasing which is given by
$\xi_n$ where $\xi$$_{\text{j}}$ $\sim$ Gumbel. This means that the purchase is dictated
by if: $V(f) + \xi_b - \xi_n > 0$. Since the difference of two Gumbel
distribution is distribution logistically, and V(f) is not
random. Knowing that the Logistic Distribution is in the location
scale family, we may consider the entire expression as: $V_i(f) \sim logit( V(f), s )$.

However, we observe some censored realizations of V. That is, we
observe valuations based on the buy orders which are non-censored
realizations, but we also observe purchases which are censored. Denote
the censored purchases by $d_n = 1$ and uncensored by: $d_n = 0$. The
likelihood function for censored data as shown by \cite{LimeBoy} is:

$$\sum_{j=1}^J \sum_{n=1}^{N_j} d_{n,j} \log( \frac{1}{4s} sech^2 ( \frac{x_{n,j} - V_j}{2s} ) ) + (1-d_{n,j}) \log ( \frac{\exp(\frac{x_{n,j} - V_j}{s})}{1+\exp(\frac{x_{n,j} - V_j}{s})} )$$

Where $V_j$ is the valuation for the box of type j, x$_{\text{n,j}}$ is the price
paid by observation n of type j, and d$_{\text{n,j}}$ is whether or not the data
at observation n of type j was censored.

We may maximize this likelihood to find the parameters that best
fit. What remains to be decided is the form of the function V. We will
have to assume a structural form for the function, and while we will
be able to test between structural forms using a validation set, there
will still be assumptions held.

\subsection{Cumulative Prospect Theory}
\label{sec-2-3}
Since one key aspect of cumulative prospect theory is that we are more
strongly motivated by losses than by gains, it is immediately obvious
that the value function of the contents of the lottery will not be
symmetric, and the easiest way to handle this will be to estimate two
separate functions, one of which is used to evaluate losses, and one
of which estimates gains. By applying a piece wise function where the
loss function is used on losses, and the gain function for gains, we
arrive at a continuous function (since both are zero at zero) which we
can use. The question of whether or not loss aversion is displayed is
one for the empirics.

Since I am following the Discrete Choice model, the decision to
purchase a create will be driven by the valuations of the crate
against the alternative which is buying nothing. Since Cumulative
prospect theory functions by looking at deviations from a reference
point, which we will use as the price of the box combined with the
costs of opening it (the key).

Following the Notation of \cite{Kahn}, we represent the valuation of the
lottery as $V(f) = \sum_{i=-m}^n \pi_i v(x_i)$ where v is a function that is
convex in losses and concave in gains. $\pi$ is a weight function that
will be defined in a later section.

The question of a reference point is commonly debated, however in this
application it is clear that the reference point for the crate is the
cost of opening such a crate, which is the sum of the "key" that must
be purchased from Valve in order to open, and the price paid for the
crate. The cost of a key is constant, and marks the direct cost of
participation, they are very often purchased directly before opening
the crate, as when you attempt to open it, you are prompted with the
price and a link to purchase a key. I believe that it is therefore
reasonable to ignore all effects time play play on the problem, and
treat the issue as if the consumer simply buys the key and crate at
the same time as he opens it. 

\subsubsection{Valuations}
\label{sec-2-3-1}
The question of how do we measure the gains of the lottery now
looms. Since each box when opened contains an item that has a
particular value to the consumer who opened it, and is unobserved, the
only thing that can be observed is the market price of the item over
time. Identification of buyer and seller valuations, even in a
simplified situation where there is only one buyer and one seller,
still requires more information than we are given. As shown in
\cite{NonParIdent}, for their identification strategy, all the bids in
the auction are required, and for the strategy shown in
\cite{PriceDataOnly}, there must exist exclusive covariates that shift
only one trader's value distribution, which are not given by the data. 

Absent an ability to identify the valuations of the specific losses
and gains in the lottery, an identifying assumption will have to be
made. I will represent valuations of the contents of the lottery as
the weighted average of the purchases made, weighted by the quantity
purchased. This is effectively the sample mean of prices purchased,
which I will assume to be average valuations of the good. 

The valuations of the i$^{\text{th}}$ possible element of the lottery will be
given by: v( p$_{\text{i}}$ - p$_{\text{l}}$ - p$_{\text{key}}$ ), where p$_{\text{i}}$ is the average price of the
i$^{\text{th}}$ element at market, p$_{\text{l}}$ is the price paid for at market for the
lottery, and p$_{\text{key}}$ is the price of the key required to open the
lottery. Depending on whether or not this difference is positive or
negative will result in different functions being used to evaluate the
valuation. Using the specification suggested by \cite{Kahn}:

$$v(x) = \begin{cases} x^\alpha \quad &x \geq 0\\ -\lambda(-x)^\alpha \quad &x < 0 \end{cases}$$

\subsubsection{Probability weighting function}
\label{sec-2-3-2}

In Cumulative Prospect theory, the cumulative mass (distribution)
function is weighted such that individuals overweight the tail
probabilities. This is especially important in this model, as there are
many high valued rare items, that if this part of the theory is
correct, heavily influence the valuation of the box, despite their
extremely low probability of occurrence.

Again, I will use the specification suggested by \cite{Kahn}, and use
the cumulative transformation function of: 

$$w(P) = \frac{ P^\delta }{( P^\delta + (1-P)^\delta )^{\frac{1}{\delta}}}$$

To define the decision weights $\pi$$_{\text{i}}$, we must first order the prospects
of the lottery in ascending order of gains. the weight $\pi$$_{\text{i}}$ then is
defined by: 
$$\pi_i = w( \sum_{j = -m}^i P(x_j)) - w( \sum_{j=-m}^{i-1} P(x_j) )$$

\subsubsection{The Valuation function}
\label{sec-2-3-3}

From these, we can create a valuation function for an individual
facing a particular lottery: 
$$V_i = \begin{cases}
(w( \sum_{j = -m}^i P(x_j)) - w( \sum_{j=-m}^{i-1} P(x_j) ))(p_i - p_l - p_{key})^\alpha \quad &(p_i - p_l - p_{key}) \geq 0 \\
-\lambda(w( \sum_{j = -m}^i P(x_j)) - w( \sum_{j=-m}^{i-1} P(x_j) ))(p_l + p_{key} - p_i)^\alpha \quad &(p_i - p_l - p_{key}) < 0 \\
\end{cases}$$


\section{Data}
\label{sec-3}
\subsection{The Data}
\label{sec-3-1}
The data is pricing data of items on the Steam Community Market for
the game Counter Strike: Global Offensive. Players in game earn items
random that they can sell on the market or open themselves. However
most rare items are earned via opening of dropped "loot boxes" that
are then opened by players via purchasing of a key. These boxes can be
earned by playing or received randomly from players who are watching
games of professionals play. The probabilities of the drops are not
known or even estimated well, as they change depending on many factors
including time. 

However, once a box has been obtained, the probability of receiving an
item is well documented as per Chinese Law. Each item has a certain
grade of rarity, for example the Ak-47 Redline has a rarity level of
Classified which means that there is a 3.2\% chance of receiving a
Classified item in the crate. All Classified items contained in the
crate have the same probability of being dropped by the crate.

However there are many variants of each item. Each item has a quality
ascribed to it, the float of the item. This describes the wear on the
item, and is distributed uniformly on the interval 0-1. On the market
the items are split into intervals: Battle-scared, well-worn,
field-tested, minimal wear and factory new. Each quality is a separate
listing on the market with a separate price. In addition to each item
having a quality type there is also a 10\% chance of each item being
labeled as StatTrak, which also distinguishes the value of a
weapon. This means that each item has 10 different variations all with
different probabilities of being obtained. Some rare items, usually
knives and gloves may have more or less variants, but the amount and
probabilities are known.

Each box contains some subset of these items that is known, and the
market value of each item at a particular time period is also known,
so the expected value, or any other modified version of a valuation of
the lottery can easily be calculated. 

\subsection{Sources of the Data}
\label{sec-3-2}
The data has been mined from the steam community market api, which
provides a purchase history for every item on the market, down to the
hour for the last thirty days and daily for the rest of the lifetime
of the item. It does not provide a record of every purchase, just the
quantity sold in that time period as well as the median price they
were sold at. Obviously this is less than ideal, but I believe it will
cause less problems than the inaccuracies introduced by the market only
working in one cent intervals.

Also available is current buy and sell orders for each item. If a
potential buyer wishes to buy on this market, he may either select a
box directly and purchase from a particular seller, or he may put
forward a buy order, which he stipulates a price, and as soon as a
seller puts an item up for sale below that price, it is sold to the
buyer, and he is charged the seller's price. This gives the
valuations of people who have not yet obtained the item directly.
However, it does not appear that there is a history available for
these items. In some ways this is beneficial because it would
impossible to determine the differences between buy orders that were
fulfilled and buy orders that were removed because of changes in the
prices of underlying assets. I have decided to treat all outstanding
buy orders as valuations in the final time period that simply are
below the market price. I will not consider the case that there are
buy orders placed and forgotten about.
% Emacs 25.3.1 (Org mode 8.2.10)
\end{document}