\documentclass[10pt]{paper}

\usepackage{amsmath}
\usepackage{float}
\usepackage{amssymb}
\usepackage{geometry}[1in,1in,1in,1in]
\usepackage{setspace}

\title{Behavioral Paper Topic}
\author{Timothy Schwieg}

\begin{document}

{ \setstretch{2}
It is often assumed that agents in economic models act with time
consistent preferences, meaning that as time passes, your preferences
do not change, this is especially important in models that can be broken
down by discretizing the time interval, and then asserting a decision
rule that is optimal in the current time period, provided that it is
followed in the following time intervals.

However violations of time inconsistency occur with regularity, and
one model that attempts to explain it is quasi-hyperbolic time
discounting. This model is relatively simple computationally, but also
allows utility streams to cross ala time inconsistent behavior. Within this
model there are sophisticated and naive time
discounters. Sophisticated time discounters are capable of
anticipating their own behavior and adjusting accordingly, while naive
ones naively choose each time period.

I wish to test if students in the Quantitative Business Tools classes
are sophisticated time discounters are not, and whether or not
allowing them to choose when they take their exams has a negative
effect on their time spent studying. I will predict if requiring
students to take out a COBA pass that forces them to take the exam at
a certain time would result in more time spent studying, and therefore
a higher grade. I intend to draw my data from the students that were enrolled in the
Quantitative Business tools classes during Fall 2017. From Canvas
there is data on when each student began the exam, as well as the window of
when the exams was open is available.

The model I wish to use is that each student faces a choice between
study,take the exam, and slack off. In a series of time intervals
until the exam each student chooses, conditioned on the fact they can always slack off after taking the exam,
studying improves their chances of getting a good grade, and that
every time interval after studying they forget a little bit. The
student then chooses a policy function to find the optimal study
time. I believe that I will have to make several structural
assumptions, especially to predict the affect of the policy change
where students would be required to use a COBA Pass. 


}
\end{document}