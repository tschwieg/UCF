\documentclass[10pt]{paper}

\usepackage{amsmath}
\usepackage{float}
\usepackage{amssymb}
\usepackage{geometry}[1in,1in,1in,1in]
\usepackage{setspace}

\title{Behavioral Paper Topic}
\author{Timothy Schwieg}

\begin{document}


{ \setstretch{2}
While often an simple computational device, expected utility theory has been
found to have many flaws that make it a poor predictive tool for analyzing
valuations of lotteries. One alternative model that has been suggested by Daniel
Kahneman is Cumulative Prospect Theory, which replaces utility with a value
function of relative losses and gains, and transforms cumulative probabilities
into subjective counter-parts.

This model requires both a value function of deviations from a reference point,
which is typically the sunk cost of entering in the lottery, and the weighting
function, which allows for over-weighting of extreme events with low
probabilities, as is often observed in lotteries.

In the video game Counter-Strike Global Offensive, players have a chance of
receiving a reward for playing: a loot box for which a key can be purchased for
a fixed amount. Upon opening this box, an item is randomly awarded to the
player. The probabilities are well known to the player, and the crate can be
sold at a market, as well as the items received from opening the crate. Almost
all transactions take place online in the Steam Community Market, which makes
much of its information public.

I wish to test to see whether or not Cumulative Prospect Theory has a better
predictive power than other models in computing the market price for these loot
boxes on the Steam Community Market. I would like to test it against other
competing models for valuations of lotteries, possibly one using arbitrage in
markets, as well as expected Utility theory. If I find that Cumulative Prospect
Theory is better at predicting the values, I wish to test how affects to the
supply of the boxes, and therefore the price of the boxes could affect total
volume sold, and therefore profit to Valve. This is conditional upon it being
possible to estimate the change in player behavior when the drop rates of the
boxes is changed, which may not be possible.

}
\end{document}