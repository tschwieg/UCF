\documentclass[12pt]{paper}

\usepackage{amssymb}
\usepackage{amsmath}
\usepackage{geometry}
\usepackage{tikz}
\usepackage{float}

\begin{document}
\section*{Question 1}

\subsection*{a}


\begin{figure}[H]
  \centering
  % Created by tikzDevice version 0.10.1 on 2018-01-25 17:39:52
% !TEX encoding = UTF-8 Unicode
\begin{tikzpicture}[x=1pt,y=1pt]
\definecolor{fillColor}{RGB}{255,255,255}
\path[use as bounding box,fill=fillColor,fill opacity=0.00] (0,0) rectangle (361.35,361.35);
\begin{scope}
\path[clip] ( 49.20, 61.20) rectangle (336.15,312.15);
\definecolor{drawColor}{RGB}{255,0,0}

\path[draw=drawColor,line width= 1.2pt,line join=round,line cap=round] (112.97,277.04) --
	(112.97,122.13) --
	(272.38,122.13);
\end{scope}
\begin{scope}
\path[clip] (  0.00,  0.00) rectangle (361.35,361.35);
\definecolor{drawColor}{RGB}{0,0,0}

\path[draw=drawColor,line width= 0.4pt,line join=round,line cap=round] ( 59.83, 61.20) -- (325.52, 61.20);

\path[draw=drawColor,line width= 0.4pt,line join=round,line cap=round] ( 59.83, 61.20) -- ( 59.83, 55.20);

\path[draw=drawColor,line width= 0.4pt,line join=round,line cap=round] (112.97, 61.20) -- (112.97, 55.20);

\path[draw=drawColor,line width= 0.4pt,line join=round,line cap=round] (166.11, 61.20) -- (166.11, 55.20);

\path[draw=drawColor,line width= 0.4pt,line join=round,line cap=round] (219.24, 61.20) -- (219.24, 55.20);

\path[draw=drawColor,line width= 0.4pt,line join=round,line cap=round] (272.38, 61.20) -- (272.38, 55.20);

\path[draw=drawColor,line width= 0.4pt,line join=round,line cap=round] (325.52, 61.20) -- (325.52, 55.20);

\node[text=drawColor,anchor=base,inner sep=0pt, outer sep=0pt, scale=  1.00] at ( 59.83, 39.60) {0};

\node[text=drawColor,anchor=base,inner sep=0pt, outer sep=0pt, scale=  1.00] at (112.97, 39.60) {1};

\node[text=drawColor,anchor=base,inner sep=0pt, outer sep=0pt, scale=  1.00] at (166.11, 39.60) {2};

\node[text=drawColor,anchor=base,inner sep=0pt, outer sep=0pt, scale=  1.00] at (219.24, 39.60) {3};

\node[text=drawColor,anchor=base,inner sep=0pt, outer sep=0pt, scale=  1.00] at (272.38, 39.60) {4};

\node[text=drawColor,anchor=base,inner sep=0pt, outer sep=0pt, scale=  1.00] at (325.52, 39.60) {5};

\path[draw=drawColor,line width= 0.4pt,line join=round,line cap=round] ( 49.20, 70.49) -- ( 49.20,277.04);

\path[draw=drawColor,line width= 0.4pt,line join=round,line cap=round] ( 49.20, 70.49) -- ( 43.20, 70.49);

\path[draw=drawColor,line width= 0.4pt,line join=round,line cap=round] ( 49.20,122.13) -- ( 43.20,122.13);

\path[draw=drawColor,line width= 0.4pt,line join=round,line cap=round] ( 49.20,173.77) -- ( 43.20,173.77);

\path[draw=drawColor,line width= 0.4pt,line join=round,line cap=round] ( 49.20,225.40) -- ( 43.20,225.40);

\path[draw=drawColor,line width= 0.4pt,line join=round,line cap=round] ( 49.20,277.04) -- ( 43.20,277.04);

\node[text=drawColor,rotate= 90.00,anchor=base,inner sep=0pt, outer sep=0pt, scale=  1.00] at ( 34.80, 70.49) {0};

\node[text=drawColor,rotate= 90.00,anchor=base,inner sep=0pt, outer sep=0pt, scale=  1.00] at ( 34.80,122.13) {2};

\node[text=drawColor,rotate= 90.00,anchor=base,inner sep=0pt, outer sep=0pt, scale=  1.00] at ( 34.80,173.77) {4};

\node[text=drawColor,rotate= 90.00,anchor=base,inner sep=0pt, outer sep=0pt, scale=  1.00] at ( 34.80,225.40) {6};

\node[text=drawColor,rotate= 90.00,anchor=base,inner sep=0pt, outer sep=0pt, scale=  1.00] at ( 34.80,277.04) {8};

\path[draw=drawColor,line width= 0.4pt,line join=round,line cap=round] ( 49.20, 61.20) --
	(336.15, 61.20) --
	(336.15,312.15) --
	( 49.20,312.15) --
	( 49.20, 61.20);
\end{scope}
\begin{scope}
\path[clip] (  0.00,  0.00) rectangle (361.35,361.35);
\definecolor{drawColor}{RGB}{0,0,0}

\node[text=drawColor,anchor=base,inner sep=0pt, outer sep=0pt, scale=  1.20] at (192.68,332.61) {\bfseries Isoquants for the Leontief};

\node[text=drawColor,anchor=base,inner sep=0pt, outer sep=0pt, scale=  1.00] at (192.68, 15.60) {h};

\node[text=drawColor,rotate= 90.00,anchor=base,inner sep=0pt, outer sep=0pt, scale=  1.00] at ( 10.80,186.67) {$\ell$};
\end{scope}
\begin{scope}
\path[clip] ( 49.20, 61.20) rectangle (336.15,312.15);
\definecolor{drawColor}{RGB}{0,0,255}

\path[draw=drawColor,line width= 1.2pt,line join=round,line cap=round] (166.11,277.04) --
	(166.11,173.77) --
	(272.38,173.77);
\definecolor{drawColor}{RGB}{0,255,0}

\path[draw=drawColor,line width= 1.2pt,line join=round,line cap=round] (219.24,277.04) --
	(219.24,225.40) --
	(272.38,225.40);
\definecolor{drawColor}{RGB}{0,0,0}

\path[draw=drawColor,line width= 0.4pt,dash pattern=on 1pt off 3pt ,line join=round,line cap=round] ( 59.83, 70.49) --
	(219.24,225.40);

\path[draw=drawColor,line width= 0.4pt,dash pattern=on 1pt off 3pt ,line join=round,line cap=round] ( 59.83,122.13) --
	(112.97,122.13);

\path[draw=drawColor,line width= 0.4pt,dash pattern=on 1pt off 3pt ,line join=round,line cap=round] (112.97, 70.49) --
	(112.97,122.13);

\node[text=drawColor,anchor=base,inner sep=0pt, outer sep=0pt, scale=  1.00] at (298.95,119.55) {$q_0 = 1$};

\node[text=drawColor,anchor=base,inner sep=0pt, outer sep=0pt, scale=  1.00] at (298.95,171.18) {$q_0 = 2$};

\node[text=drawColor,anchor=base,inner sep=0pt, outer sep=0pt, scale=  1.00] at (298.95,222.82) {$q_0 = 3$};
\end{scope}
\end{tikzpicture}

  \caption{Level Sets of a Leontief Production Function}
\end{figure}


$\alpha$ and $\beta$ are technology parameters that control how much of each input to
reach a certain amount of output. To reach q level of output, one needs
$\frac{q}{\alpha} h$ and $\frac{q}{\beta} \ell$. We may note that this production function
has constant returns to scale, so it has a linear scale effect.

\subsection*{b}


$$C( q, w,s,\alpha,\beta) = \min_{h,\ell}   hw + \ell s \quad \text{ s.t. }\min\{ \alpha h, \beta \ell \} \geq q$$
Since w and s are input prices, we will assume they are positive, so the
constraint will bind. Since the constraint binds, we know that: $\min( \alpha h, \beta \ell )
= q$ and that $\alpha h = \beta \ell = q$. If it were otherwise, the firm could lower the usage
of either h or $\ell$ and minimize the objective function without effecting the
constraint. Therefore: $h = \frac{q}{\alpha}$ and $\ell = \frac{q}{\beta}$. The cost
function is therefore: $$C( \alpha, \beta, w, s, q ) = \frac{wq\beta+ sq\alpha}{\alpha\beta} = q \frac{w\beta +
  s\alpha}{\alpha\beta}$$ \newline
The marginal cost of this function is the derivative with respect to
q, and is: $$m( \alpha, \beta, w, s ) =  \frac{w\beta + s\alpha}{\alpha\beta}$$


\begin{figure}[H]
  \centering
  % Created by tikzDevice version 0.10.1 on 2018-01-25 17:39:52
% !TEX encoding = UTF-8 Unicode
\begin{tikzpicture}[x=1pt,y=1pt]
\definecolor{fillColor}{RGB}{255,255,255}
\path[use as bounding box,fill=fillColor,fill opacity=0.00] (0,0) rectangle (180.67,180.67);
\begin{scope}
\path[clip] ( 49.20, 61.20) rectangle (155.47,131.47);
\definecolor{drawColor}{RGB}{255,0,0}

\path[draw=drawColor,line width= 1.2pt,line join=round,line cap=round] ( 53.14, 63.80) --
	(131.86,121.64);
\end{scope}
\begin{scope}
\path[clip] (  0.00,  0.00) rectangle (180.67,180.67);
\definecolor{drawColor}{RGB}{0,0,0}

\path[draw=drawColor,line width= 0.4pt,line join=round,line cap=round] ( 53.14, 61.20) -- (151.54, 61.20);

\path[draw=drawColor,line width= 0.4pt,line join=round,line cap=round] ( 53.14, 61.20) -- ( 53.14, 55.20);

\path[draw=drawColor,line width= 0.4pt,line join=round,line cap=round] ( 72.82, 61.20) -- ( 72.82, 55.20);

\path[draw=drawColor,line width= 0.4pt,line join=round,line cap=round] ( 92.50, 61.20) -- ( 92.50, 55.20);

\path[draw=drawColor,line width= 0.4pt,line join=round,line cap=round] (112.18, 61.20) -- (112.18, 55.20);

\path[draw=drawColor,line width= 0.4pt,line join=round,line cap=round] (131.86, 61.20) -- (131.86, 55.20);

\path[draw=drawColor,line width= 0.4pt,line join=round,line cap=round] (151.54, 61.20) -- (151.54, 55.20);

\node[text=drawColor,anchor=base,inner sep=0pt, outer sep=0pt, scale=  1.00] at ( 53.14, 39.60) {0};

\node[text=drawColor,anchor=base,inner sep=0pt, outer sep=0pt, scale=  1.00] at ( 72.82, 39.60) {1};

\node[text=drawColor,anchor=base,inner sep=0pt, outer sep=0pt, scale=  1.00] at ( 92.50, 39.60) {2};

\node[text=drawColor,anchor=base,inner sep=0pt, outer sep=0pt, scale=  1.00] at (112.18, 39.60) {3};

\node[text=drawColor,anchor=base,inner sep=0pt, outer sep=0pt, scale=  1.00] at (131.86, 39.60) {4};

\node[text=drawColor,anchor=base,inner sep=0pt, outer sep=0pt, scale=  1.00] at (151.54, 39.60) {5};

\path[draw=drawColor,line width= 0.4pt,line join=round,line cap=round] ( 49.20, 63.80) -- ( 49.20,121.64);

\path[draw=drawColor,line width= 0.4pt,line join=round,line cap=round] ( 49.20, 63.80) -- ( 43.20, 63.80);

\path[draw=drawColor,line width= 0.4pt,line join=round,line cap=round] ( 49.20, 78.26) -- ( 43.20, 78.26);

\path[draw=drawColor,line width= 0.4pt,line join=round,line cap=round] ( 49.20, 92.72) -- ( 43.20, 92.72);

\path[draw=drawColor,line width= 0.4pt,line join=round,line cap=round] ( 49.20,107.18) -- ( 43.20,107.18);

\path[draw=drawColor,line width= 0.4pt,line join=round,line cap=round] ( 49.20,121.64) -- ( 43.20,121.64);

\node[text=drawColor,rotate= 90.00,anchor=base,inner sep=0pt, outer sep=0pt, scale=  1.00] at ( 34.80, 63.80) {0};

\node[text=drawColor,rotate= 90.00,anchor=base,inner sep=0pt, outer sep=0pt, scale=  1.00] at ( 34.80, 78.26) {2};

\node[text=drawColor,rotate= 90.00,anchor=base,inner sep=0pt, outer sep=0pt, scale=  1.00] at ( 34.80, 92.72) {4};

\node[text=drawColor,rotate= 90.00,anchor=base,inner sep=0pt, outer sep=0pt, scale=  1.00] at ( 34.80,107.18) {6};

\node[text=drawColor,rotate= 90.00,anchor=base,inner sep=0pt, outer sep=0pt, scale=  1.00] at ( 34.80,121.64) {8};

\path[draw=drawColor,line width= 0.4pt,line join=round,line cap=round] ( 49.20, 61.20) --
	(155.47, 61.20) --
	(155.47,131.47) --
	( 49.20,131.47) --
	( 49.20, 61.20);
\end{scope}
\begin{scope}
\path[clip] (  0.00,  0.00) rectangle (180.67,180.67);
\definecolor{drawColor}{RGB}{0,0,0}

\node[text=drawColor,anchor=base,inner sep=0pt, outer sep=0pt, scale=  1.20] at (102.34,151.93) {\bfseries Total Cost};

\node[text=drawColor,anchor=base,inner sep=0pt, outer sep=0pt, scale=  1.00] at (102.34, 15.60) {q};

\node[text=drawColor,rotate= 90.00,anchor=base,inner sep=0pt, outer sep=0pt, scale=  1.00] at ( 10.80, 96.34) {C(q)};
\end{scope}
\end{tikzpicture}
\begin{tikzpicture}[x=1pt,y=1pt]
\definecolor{fillColor}{RGB}{255,255,255}
\path[use as bounding box,fill=fillColor,fill opacity=0.00] (0,0) rectangle (180.67,180.67);
\begin{scope}
\path[clip] ( 49.20, 61.20) rectangle (155.47,131.47);
\definecolor{drawColor}{RGB}{0,0,255}

\path[draw=drawColor,line width= 1.2pt,line join=round,line cap=round] ( 53.14, 96.34) --
	(151.54, 96.34);
\end{scope}
\begin{scope}
\path[clip] (  0.00,  0.00) rectangle (180.67,180.67);
\definecolor{drawColor}{RGB}{0,0,0}

\path[draw=drawColor,line width= 0.4pt,line join=round,line cap=round] ( 53.14, 61.20) -- (151.54, 61.20);

\path[draw=drawColor,line width= 0.4pt,line join=round,line cap=round] ( 53.14, 61.20) -- ( 53.14, 55.20);

\path[draw=drawColor,line width= 0.4pt,line join=round,line cap=round] ( 77.74, 61.20) -- ( 77.74, 55.20);

\path[draw=drawColor,line width= 0.4pt,line join=round,line cap=round] (102.34, 61.20) -- (102.34, 55.20);

\path[draw=drawColor,line width= 0.4pt,line join=round,line cap=round] (126.94, 61.20) -- (126.94, 55.20);

\path[draw=drawColor,line width= 0.4pt,line join=round,line cap=round] (151.54, 61.20) -- (151.54, 55.20);

\node[text=drawColor,anchor=base,inner sep=0pt, outer sep=0pt, scale=  1.00] at ( 53.14, 39.60) {0};

\node[text=drawColor,anchor=base,inner sep=0pt, outer sep=0pt, scale=  1.00] at ( 77.74, 39.60) {1};

\node[text=drawColor,anchor=base,inner sep=0pt, outer sep=0pt, scale=  1.00] at (102.34, 39.60) {2};

\node[text=drawColor,anchor=base,inner sep=0pt, outer sep=0pt, scale=  1.00] at (126.94, 39.60) {3};

\node[text=drawColor,anchor=base,inner sep=0pt, outer sep=0pt, scale=  1.00] at (151.54, 39.60) {4};

\path[draw=drawColor,line width= 0.4pt,line join=round,line cap=round] ( 49.20, 63.80) -- ( 49.20,128.87);

\path[draw=drawColor,line width= 0.4pt,line join=round,line cap=round] ( 49.20, 63.80) -- ( 43.20, 63.80);

\path[draw=drawColor,line width= 0.4pt,line join=round,line cap=round] ( 49.20, 80.07) -- ( 43.20, 80.07);

\path[draw=drawColor,line width= 0.4pt,line join=round,line cap=round] ( 49.20, 96.34) -- ( 43.20, 96.34);

\path[draw=drawColor,line width= 0.4pt,line join=round,line cap=round] ( 49.20,112.60) -- ( 43.20,112.60);

\path[draw=drawColor,line width= 0.4pt,line join=round,line cap=round] ( 49.20,128.87) -- ( 43.20,128.87);

\node[text=drawColor,rotate= 90.00,anchor=base,inner sep=0pt, outer sep=0pt, scale=  1.00] at ( 34.80, 63.80) {0};

\node[text=drawColor,rotate= 90.00,anchor=base,inner sep=0pt, outer sep=0pt, scale=  1.00] at ( 34.80, 80.07) {1};

\node[text=drawColor,rotate= 90.00,anchor=base,inner sep=0pt, outer sep=0pt, scale=  1.00] at ( 34.80, 96.34) {2};

\node[text=drawColor,rotate= 90.00,anchor=base,inner sep=0pt, outer sep=0pt, scale=  1.00] at ( 34.80,112.60) {3};

\node[text=drawColor,rotate= 90.00,anchor=base,inner sep=0pt, outer sep=0pt, scale=  1.00] at ( 34.80,128.87) {4};

\path[draw=drawColor,line width= 0.4pt,line join=round,line cap=round] ( 49.20, 61.20) --
	(155.47, 61.20) --
	(155.47,131.47) --
	( 49.20,131.47) --
	( 49.20, 61.20);
\end{scope}
\begin{scope}
\path[clip] (  0.00,  0.00) rectangle (180.67,180.67);
\definecolor{drawColor}{RGB}{0,0,0}

\node[text=drawColor,anchor=base,inner sep=0pt, outer sep=0pt, scale=  1.20] at (102.34,151.93) {\bfseries Marginal Cost};

\node[text=drawColor,anchor=base,inner sep=0pt, outer sep=0pt, scale=  1.00] at (102.34, 15.60) {q};

\node[text=drawColor,rotate= 90.00,anchor=base,inner sep=0pt, outer sep=0pt, scale=  1.00] at ( 10.80, 96.34) {m(q)};
\end{scope}
\end{tikzpicture}

  \caption{Total and Marginal Cost}
\end{figure}

\section*{Question 2}

The purpose of game theory is to be able to study strategic interaction between
agents when all agents have some power. The two main branches of Game Theory are
strategic and coalitional games. In Strategic Games player's available moves are
specified, and optimal behavior is derived. In Coalitional games, optimal
behavior is described. All games feature:
\begin{itemize}
\item Players: $P$
\item Rules: $R$
\item Information structure stating what each player knows: $I$
\item Set of Strategies available to players: $S_i$
\item Payoff functions for strategies: $U_i$
\item Some Notion of equilibrium
\end{itemize}

Two notions of equilibrium used to solve games are Nash-Equilibrium, where each
player is playing the best-response to opponents' best response functions; and
Dominant equilibrium, where a player plays a strategy irrespective of the
actions taken by the other players.

One example of a game is a student who wishes to ask questions to a certain
professor, but if he asks a dumb question, the professor will chastise him in
front of his colleagues. He must strategically estimate how the professor will
react to his question based on prior knowledge to decide whether or not he
should ask it, while the professor must decide whether or not to harshly respond
in order to either minimize or maximize questions depending on his preferences
of class. In this case $P = \{ $Student, Professor$\}$, the Rules are that the
student may ask any question, and the professor may respond in any way. The
student may continue to ask questions as long as class is still in session.
\newline We assume that the professor has full knowledge of the subject, so he has perfect
information about if the question is stupid or not, while the student only has
partial information, but with each question successfully asked gains
information. The student has no knowledge of how the professor will respond, and
the professor knows that a certain amount of belittling will dissuade the
student from asking questions, but he does not know the threshold.\newline Therefore the
set of strategies for the student is a distribution of questions where the more
informative questions are more likely to be viewed by the professor as stupid,
and no questions, for when the student has been sufficiently chastised. This is
crossed with itself for a countably infinite number of times. The professor's
strategy is a continuum of how well to answer the question (encouraging
questions and learning at the cost of time) and how much to berate the student
(discouraging questions and increasing time available for lecture.)\newline The student
faces an increasing utility for information gained, and some negative quantity
for when he is berated by the professor. The professor may have an increasing
utility for increasing student knowledge, or any other utility function
depending on his desires while teaching. \newline It is unclear whether or not this
game would lead to a Bayes-Nash equilibrium because of very little being stated
about the payoffs of the Professor, and more structure may need to be imposed to
enforce an equilibrium arises.
\newline \newline
Another example would be deciding where to go to lunch. The players in the game
are all the other people seeking lunch, and everyone dislikes waiting in a line,
so there is a negative effect caused by others going to the same restaurant as
you. The set of players is everyone who is seeking lunch.\newline The rules are that
they must wait in line when they arrive behind everyone else who has already
arrived. \newline The information structure is such that each person knows when they
leave for lunch, and the distribution of when everyone is getting lunch, but not
specific times when each person has gone for lunch. \newline The payoff functions are
such that each person has preferences over restaurants and in waiting in line,
and the combination of this forms their utility for the lunch.

\section*{Question 3}

\subsection*{a.}

By inspection, we may note that child A finds confess to be a dominant strategy,
since $-1 > -2$ and $0 > -2$. By the same logic, child B finds confession to be
a dominant strategy, so the equilibrium will be both confess and face payoffs of
-1 apiece. This equilibrium is a Dominant Equilibrium, as both chose strategies
that were best for them regardless of the other's choice.

\subsection*{b}

If child A knew B would confess, his best response would be to confess as well,
and if he knew that B would be silent, his best response is to be silent. By
symmetry, the best response for B is the same. We can clearly see that both
children have a best response of being silent when the other is silent, and to
confess when the other confesses. Thus there are two Nash Equilibria, both
confessing, and both being silent.

\section*{Question 4}

Extensive form games are games that allow for sequential play, leading to
dynamics. This element allows them to accommodate incomplete
information. This is managed through the use of the decision tree, which allows
for an easy representation of not only their available moves at each stage, but
what information they have been exposed to and can therefore act upon.

One phenomena involving incomplete information is a private-values auction,
where the buyers all have private valuations of an object that they are very
unlikely to disclose, especially to a seller. This can cause the participants to
lie about their valuations, particularly if they have high private valuations
and they are forced to pay their bid.

Another phenomena is a game of poker, where the cards held by each player are
private information, and the only information given is by bets placed at each
round as public information (flop turn river) is given out. Players have an
incentive to lie in order to give the false idea that they have a better or
worse hand than they do.


\end{document}