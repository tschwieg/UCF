% Created 2018-04-13 Fri 16:59
\documentclass[bigger]{beamer}
\usepackage[utf8]{inputenc}
\usepackage[T1]{fontenc}
\usepackage{fixltx2e}
\usepackage{graphicx}
\usepackage{longtable}
\usepackage{float}
\usepackage{wrapfig}
\usepackage{rotating}
\usepackage[normalem]{ulem}
\usepackage{amsmath}
\usepackage{textcomp}
\usepackage{marvosym}
\usepackage{wasysym}
\usepackage{amssymb}
\usepackage{hyperref}
\tolerance=1000
\usepackage{minted}
\usepackage{mathtools}
\usepackage{tikz}
\makeatletter
\newcommand\mathcircled[1]{%
\mathpalette\@mathcircled{#1}%
}
\newcommand\@mathcircled[2]{%
\tikz[baseline=(math.base)] \node[draw,circle,inner sep=1pt] (math) {$\m@th#1#2$};%
}
\makeatother
\usetheme{Montpellier}
\author{Timothy Schwieg}
\date{\today}
\title{Risk, Ambiguity, and the Savage Axioms}
\hypersetup{
  pdfkeywords={},
  pdfsubject={},
  pdfcreator={Emacs 25.3.1 (Org mode 8.2.10)}}
\begin{document}

\maketitle


\section{Uncertainty}
\label{sec-1}
\begin{frame}[label=sec-1-1]{How much uncertainty is there?}
\begin{itemize}
\item Even though Knight claims uncertainty dominates, people tend to
behave "as if" they have numerical probabilities assigned to events.
\item Shackle's Cricket Match
\item Can we provide odds we are willing to back?
\end{itemize}
\end{frame}

\begin{frame}[label=sec-1-2]{}
\begin{itemize}
\item "The race is not always to the swift nor the battle to the strong,
but that's the way to bet." - Damon Runyon
\item Do these bets reveal anything about our "hidden probabilities?"
\item If this were possible, we could turn these uncertainties that people
were willing to bet on, and transform them into something alike
risk.
\end{itemize}
\end{frame}

\begin{frame}[label=sec-1-3]{Enter Savage}
\begin{itemize}
\item Savage claims that for someone who is "rational" all uncertainties
can be reduced into risks.
\item A willingness to bet on 2:1 odds means a belief in a probability of
$\frac{1}{3}$
\item Must still somehow untangle probabilities from preferences, but in a
controlled setting this is feasible.
\item What does "rational" mean?
\end{itemize}
\end{frame}

\begin{frame}[label=sec-1-4]{What can we infer from behavior?}
\begin{itemize}
\item What does indifference between two gambles imply?
\item Equal probabilities, given you assigned probabilities.
\item May just be attempting to mimimax - this only reflects they have the
same worst-case.
\item Seek an operator $\mathcircled { \geq }$ - Qualitative Probability
\end{itemize}
\end{frame}

\begin{frame}[label=sec-1-5]{What the operator gives us:}
\begin{itemize}
\item $\mathcircled { \geq }$ is a complete ordering among events
\item If an event $\alpha$ is more probable than $\beta$, then $\bar{\alpha}$ is less
probable than $\bar{\beta}$.
\item If $\alpha$ and $\gamma$ are mutually exclusive and so are $\beta$ and $\gamma$, then if $\alpha$ is
more probable than $\beta$, $\alpha \cup \gamma$ is more probable than $\beta \cup \gamma$
\item Under the Savage axioms, $\mathcircled { \geq }$ holds these
properties.
\end{itemize}
\end{frame}

\begin{frame}[label=sec-1-6]{The Savage axioms}
\begin{itemize}
\item Gambles have a complete ordering
\item If two gambles have the same payoff, its value is irrelevant.
\item Dominated Actions are always rejected - "noncontroversial"
\item Choice in a gamble is independent of the relative magnitudes of the
rewards, only the ranking.

Size of the prize doesn't affect choice.
\end{itemize}
\end{frame}

\begin{frame}[label=sec-1-7]{What happens when these are not fulfilled?}
\begin{itemize}
\item We cannot infer probabilities from actions
\item There is no VNM Utility function that we can apply.
\item These are not met in a particular class of situations
\item These situations are where ambiguity rules.
\end{itemize}
\end{frame}

\section{Do the Axioms Apply?}
\label{sec-2}
\begin{frame}[label=sec-2-1]{Experiment Time}
\begin{itemize}
\item Which would you prefer to bet on?
\end{itemize}
\begin{center}
\begin{tabular}{ll}
Box I & 10 Red M\&Ms, 10 Green M\&Ms\\
Box II & 20 M\&Ms All of which are Red or Green\\
\end{tabular}
\end{center}
\begin{itemize}
\item Red M\&M or Green M\&M from Box I?
\item Red M\&M or Green M\&M from Box II?
\item Red M\&M from Box I or Red from BoxII?
\item Green from Box I or Green from BoxII?
\end{itemize}
\end{frame}

\begin{frame}[label=sec-2-2]{Honey I broke the axioms}
\begin{itemize}
\item Preferring Box I to Box II in both violates the savage axioms.
\item If you prefer Box I Red, then you must view it as more probable, and
therefore Box I Green as less probable than Box II Green.
\item However you also prefer Box II Green to Box I Green, and we have a
contradiction.
\end{itemize}
\end{frame}

\begin{frame}[label=sec-2-3]{Do we really have total ignorance?}
\begin{itemize}
\item Lets take a few samples out of Box II.
\item Have your preferences for betting changed?
\item This is typical, and it only changes willingness to bet slightly.
\end{itemize}
\end{frame}

\begin{frame}[label=sec-2-4]{Another Experiment}
\begin{itemize}
\item There are 20 known blue M\&Ms, and forty M\&Ms that are all red or
green.
\end{itemize}
\begin{center}
\begin{tabular}{ll}
Bet on Blue & Bet on Red\\
Bet on Blue or Green & Bet on Red or Green\\
\end{tabular}
\end{center}
\begin{itemize}
\item When the ambiguity is removed by Red or Green being an option, it is
suddenly the preferred case.
\item Sophisticated individuals still continue to violate the axioms even
on reflection with the notion they are violating them.
\end{itemize}
\end{frame}

\section{Ambiguity}
\label{sec-3}
\begin{frame}[label=sec-3-1]{Standard uncertainty behaviors}
\begin{itemize}
\item Individuals are not minimaxing purely.
\item Nor are they maximizing any weighted average of the best and worst
case.
\item They aren't even minimaxing regret, as these examples are designed
so that their regret would be identical.
\item Yet these choices are fairly obvious and intuitive.
\end{itemize}
\end{frame}

\begin{frame}[label=sec-3-2]{Ambiguity Aversion}
\begin{itemize}
\item We have some information about the problem, but we just aren't sure
how good our information is.
\item However we aren't "completely ignorant" so common techniques for
handling uncertainty don't apply either.
\item Limited to distributions $(\frac{1}{3}, \lambda, \frac{2}{3} - \lambda)$.
\item No real knowledge of which of these distributions is more "likely"
\item This is different from the uninformed prior!
\end{itemize}
\end{frame}

\begin{frame}[label=sec-3-3]{Further Problems}
\begin{itemize}
\item Even if an individual could assign relative weights to each possible
distribution and apply a prior, he does not know how useful the data
is.
\item Information can still be ambiguous, for example hearing about a
players' FG\% in basketball may inform you on their chances in the
playoffs this year more or less depending on your knowledge.
\item This cannot be expressed in terms of likelihoods.
\end{itemize}
\end{frame}

\begin{frame}[label=sec-3-4]{Where does ambiguity apply?}
\begin{itemize}
\item Where information can often be unreliable.
\item New processes. (Vegas Games rely on the same stochastic devices).
\item Returning to the Cricket Game:

You would be certain to bet on who bats first

How certain are you on betting on who wins the game?
\end{itemize}
\end{frame}

\begin{frame}[label=sec-3-5]{A possible Computational Device}
\begin{itemize}
\item It is possible that maybe there is an expected distribution (prior),
but since the individual is not certain that he is correct, he
weights it against the worst possible case.
\item Ellsberg uses a linear combination for simplicity, but it could be
more complex.
\item Provides nothing more than a heuristic.
\item No formal model presented.
\end{itemize}
\end{frame}


\begin{frame}[label=sec-3-6]{How does this predict behavior?}
\begin{itemize}
\item When we are less sure of our predictions, we become more
conservative.
\item A bet on known things is preferred to a bet on unknown odds.
\item This doesn't mean that people are not acting optimally, nor are they
being "irrational"
\end{itemize}
\end{frame}
% Emacs 25.3.1 (Org mode 8.2.10)
\end{document}