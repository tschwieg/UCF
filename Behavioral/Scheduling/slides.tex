\documentclass[grey,handout]{beamer}
%\usetheme{Pittsburgh}
\usetheme{Montpellier}
\usepackage{amsmath}

\renewcommand{\frametitle}[1]{\begin{center}\textbf{#1}\end{center}}

\def\dd{{\rm d}}
\def\E{\mathbb{E}}
\def\BigO{{\cal O}}

\begin{document}
\title{Scheduling}

\author{Timothy M Schwieg}
\date{3 March 2018}

\begin{frame}
  \titlepage
\end{frame}

\begin{frame}
\frametitle{ROADMAP OF SEMINAR}
  \begin{enumerate}[<+->]
  \item Introduction to Scheduling
  \item Parts of each algorithm
  \item Earliest Due Date
  \item Shortest Processing Time
  \item Moore's Algorithm
  \item Preemption costs
  \end{enumerate}
\end{frame}

\section{Introduction To Scheduling}

\begin{frame}
  \frametitle{History and Focus}
  \begin{itemize}[<+->]
  \item 1874: Frederick Taylor and Henry Gatt coin the terms scientific
    management. Make public schedules but do not solve any problems
  \item 1954: Selmer Johnson while working for RAND publishes a
    solution for solving a two-machine process of book binding.
  \item Focus upon a single machine. However the order we do it is
    irrelevant to the time taken to solve the problem without order constraints.
  \end{itemize}
\end{frame}



\begin{frame}
  \frametitle{Complications}
  \begin{itemize}
  \item Differing weights between tasks - usually managable
  \item Release times - Destroys Tractibility
  \item Precendence Constraints
  \item Preemption
  \item Online Algorithms
  \end{itemize}
\end{frame}

\section{Earliest Due Date}

\begin{frame}
  \frametitle{Earliest Due Date - Minimize Maximum Lateness}
  \begin{itemize}
  \item Cost Function: $c_j(t) = (t - d_j)^+$
  \item Find a job k such that: $c_k(T) = \min \{ (T - d_j)^+ \} T =
      \sum_{n=0}^N p_n$
  \item This is equivalent to taking the task with the maximum due
        date.
  \item Complexity is $\BigO ( N \log( N ) )$, Independent of
          Processing Time.
  \end{itemize}
\end{frame}

\begin{frame}
  \frametitle{Precedence Constraints}
  \begin{itemize}
  \item We can't pick any job to go last, only since some jobs must
    precede others. Define J to be the sets of jobs that do not have
    to precede others.
  \item Find a job $k \in J$ where $c_k(T) = \min_{j \in J} \{ (T - d_j
      )^+ \}$
  \item Schedule that job k to be last, recalculate the set J to
        be the jobs not including k that can go last and repeat.
  \item This is $\BigO ( N^2 )$ and processing time does not matter
  \end{itemize}
\end{frame}

\begin{frame}
  \frametitle{ Preemption }
  \begin{itemize}
  \item Machine can switch to tasks for free, so only time machine is
    idle is when nothing can be done
  \item Seek blocks where there is no downtime.
  \item Sort by release dates, and use this to break up the tasks into
    blocks. This can be done in $\BigO ( N )$ time.
  \item Find a job l in each block where $c_l(t(B) ) = \min_{b \in B}
    \{ (t(B) - d_b)^+ \}$
  \item Schedule this job last, then continue scheduling the block
    with the same policy until it is fully scheduled.
  \item This runs in $\BigO( N^2 )$ runtime, completing at most N-1 preemptions
  \end{itemize}
\end{frame}

\section{Shortest Processing Time}

\begin{frame}
  \frametitle{Shortest Processing Time - Minimizing Outstanding Tasks}
  \begin{itemize}
  \item We want to get things done as quickly as possible, so start
    with the tasks we can do the quickest
  \item Sort the tasks based on processing time, and complete in
    ascending order
  \item This can be computed in $\BigO ( N \log( N ) )$ run-time as it
    only really requires a sort.
  \end{itemize}
\end{frame}

\begin{frame}
  \frametitle{ Complications}
  \begin{itemize}
  \item Weights:
    \begin{itemize}
    \item Greedy Algorithm pays off - we can sort it based on:
      $\frac{p_j}{w_j}$.
    \item this is still $\BigO( N \log( N ) )$
    \end{itemize}
  \item Pre-emption:
    \begin{itemize}
    \item Switch at a start time when a new task is available, if it
      has a lower processing time, otherwise continue as before.
    \end{itemize}
  \item Precedence Relations:
    \begin{itemize}
    \item This becomes NP-hard, and it is pretty difficult to break
      the computational problem. Online algorithm is one method.
    \end{itemize}
  \end{itemize}
\end{frame}

\begin{frame}
  \frametitle{Online Alogrithm}
  \begin{itemize}
  \item Greedy Algorithm doesn't work here, as we can be stuck doing something
    when it is not optimal.
  \item Change suggested by: Anderson and Potts 2004 - When a new job appears,
    compare it to the current job by minimizing $\frac{p_j}{w_j}$, and
    scheduling job j to start at: $\max \{ t, p_j \}$.
  \item This ensures the algorithm is 2-competitive, at worst case twice as bad
    as the offline version. This is the best any online algorithm can do.
  \end{itemize}
\end{frame}

\section{Moore's Algorithm}

\begin{frame}
  \frametitle{Moore's Algorithm - Minimizing Number of Late Tasks}
  \begin{itemize}
    \item No penalty for how late a task is - No longer seeking a permutation,
      only an optimal subset of the tasks
    \item This leaves: $\sum_{n=0}^N \binom{ N}{n} = 2^N$ possible partitions.
    \item Want to use Dynamic Programming to turn this into $\BigO( N \log(N) )$
      or $\BigO( N^2 )$.
    \item Integer Program: $x_n$ denotes if job n, ordered by earliest due date,
      is late or not.
    \item $\max \sum_{n=1}^N w_n x_n$ subject to: $\sum_{i=1}^j p_i x_i \leq d_j \quad \forall j 1 \leq j
      \leq n$
 \end{itemize}
\end{frame}

\begin{frame}
  \frametitle{Dynamic Programming Algorithm}
  \begin{itemize}
    \item Create a function $F_j(t)$ which tells us the optimal subset of first
      j jobs to complete in time t.
    \item We seek $F_N(T)$ where T is the time
      required to do all the jobs we choose to do.
    \item Base Case: $F_0(0) = 0, F_0(t) = \infty, t > 0$.
    \item If we add a new job: $F_j(t)= F_{j-1}(t) + w_j$
    \item Don't add job: $F_j(t) = F_{j-1}(t - p_j)$
  \end{itemize}
\end{frame}

\begin{frame}
  \frametitle{Dynamic Programing Algorithm Conintued}
  \begin{itemize}
  \item Combine these two into a decision rule: $$F_j(t) = 
    \begin{cases} \min \{ F_{j-1}(t-p_j), F_{j-1}(t) + w_j \} & \text{ if } 0 \leq t \le
      d_j\\
      F_j(d_j) & \text{ if } d_j \le t \le T\\
    \end{cases}$$
  \item Algorithm: For j in 1:N do:\newline
    for t in 0:T do:\newline
    Update $F_j(t)$
  \item Optimal value is then $F_n(d_n)$ in $\BigO( N^2)$ time.
  \end{itemize}
\end{frame}

\begin{frame}
  \frametitle{Complications}
  \begin{itemize}
  \item Symmetry of the constraints was used to build the program, not possible
    once precedence constraints and release dates are input.
  \item This returns to the integer programs we saw in Operations
    Research. These problems are NP-Hard.
  \item If preemption is allowed, the problem may be solved in Pseudo-Polynomial
    time. See Lawler (1990) for a $\BigO( N K^2 W^2)$ algorithm.
  \end{itemize}
\end{frame}

\section{Preemption Costs}

\begin{frame}
  \frametitle{Preemption Costs}
  \begin{itemize}
  \item It is not always costs less to switch between tasks, but this is
    difficult to compute tractably. This price is called a context-switch.
  \item The time spent switching between tasks is called meta-work.
  \item This creates a trade-off between Responsiveness, how quickly a machine
    changes tasks, and Throughput, how much a machine can accomplish.
  \item This is a vector optimization problem, and choices on the Pareto
    Frontier must be made in the context of the specific machine. E.x. Computers
    limit responsiveness of the mouse to the smallest interval the eye can
    detect, but may increase it during a video game.
  \end{itemize}
\end{frame}

\section{Summary and Conclusions}

\begin{frame}
  \frametitle{Summary and Conclusions}
  \begin{itemize}[<+->]
  \item What objective you choose determines what order tasks should
    be done, and is dependent on your needs for the machine
  \item Adding release dates or precedence constraints often has the
    algorithm exit the realm of tractable problems.
  \item Online algorithms and Preemption allow for a return back to
    Tractibility.
  \item Preemption is not always free, and context-switching too often
    can lower throughput, or cause thrashing at worst case.
  \end{itemize}
\end{frame}

\end{document}