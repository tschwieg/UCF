% Created 2018-03-30 Fri 20:22
\documentclass[bigger]{beamer}
\usepackage[utf8]{inputenc}
\usepackage[T1]{fontenc}
\usepackage{fixltx2e}
\usepackage{graphicx}
\usepackage{longtable}
\usepackage{float}
\usepackage{wrapfig}
\usepackage{rotating}
\usepackage[normalem]{ulem}
\usepackage{amsmath}
\usepackage{textcomp}
\usepackage{marvosym}
\usepackage{wasysym}
\usepackage{amssymb}
\usepackage{hyperref}
\tolerance=1000
\usetheme{Montpellier}
\author{Timothy Schwieg}
\date{\today}
\title{The Signal and the Noise}
\hypersetup{
  pdfkeywords={},
  pdfsubject={},
  pdfcreator={Emacs 25.1.1 (Org mode 8.2.10)}}
\begin{document}

\maketitle

\section{Introduction}
\label{sec-1}

\begin{frame}[label=sec-1-0-1]{Prediction}
\begin{itemize}
\item Separate Signal from the Noise
\item We're Mostly bad at this. Very Bad
\item Information Overload  Printing Press
\item Productivity Paradox  Information Age
\end{itemize}
\end{frame}

\section{Prediction - Where it works and where it doesn't}
\label{sec-2}
\subsection{The Great Recession}
\label{sec-2-1}

\begin{frame}[label=sec-2-1-1]{Rating Agencies}
\begin{itemize}
\item Forecasted the risk of CDOs at 0.12\%
\item Actual Default Rate at 28\%
\item Rating Quality had little effect on their profits
\item Benefited almost solely from rating securities
\item Simulated under assumptions and missed risk they were taking
\end{itemize}
\end{frame}


\begin{frame}[label=sec-2-1-2]{Risk and Uncertainty}
\begin{itemize}
\item Depending on the Dependence structure of the underlying assets,
bundled securities can be much less or more risky than under
assumptions such as independence.
\item The Nature of the dependence is not known - Knightian Uncertainty
\item Rating agencies claimed this uncertainty was risk - Moody's
increased their estimate of default by 50\%. Which did nothing to
mollify.
\end{itemize}
\end{frame}

\begin{frame}[label=sec-2-1-3]{Prediction in the Housing Bubble}
\begin{itemize}
\item Homeowners assumed the prices would continue to rise -ignoring the
bubble
\item Banks and rating agencies did not understand the system risk present
\item Economists did not understand how the leveraged bank sector was tied
to the housing market
\item Policy makers were overconfident in the rate of Recovery of the
Economy - basing it falsely on recessions instead of financial
crises.
\end{itemize}
\end{frame}

\begin{frame}[label=sec-2-1-4]{Understanding Your Sample}
\begin{itemize}
\item Should you Drunk Drive or Call Uber?
\item Can we trust Rating Agencies to know how to rate new assets?
\item You should always use your sample to avoid false confidence
\end{itemize}
\end{frame}

\subsection{Political Predictions}
\label{sec-2-2}

\begin{frame}[label=sec-2-2-1]{McLaughlin Show}
\begin{itemize}
\item Predictions occur at the final part of the episode
\item Regardless of Political Affiliation, no guess is any better than 50\%
\item However the Predictions are structured with 75z\% of them being
completely false or Completely true
\end{itemize}
\end{frame}

\begin{frame}[label=sec-2-2-2]{Political Scientists}
\begin{itemize}
\item Almost nobody correctly predicted the Collapse of the Soviet Union
\item Required piecing together structural problems with Soviet union and
Gorbachev's motivations for reform.
\item Poor estimates of Soviet GDP and its decline
\item However this isn't unique to the Soviet Union - most political
scientists are terrible at predicting anything.
\item In fact, the ones that speak to the media do worse than ones that don't
\end{itemize}
\end{frame}

\begin{frame}[label=sec-2-2-3]{Foxes vs Hedgehogs}
\begin{itemize}
\item "The Fox knows many little things, but the hedgehog knows one big
thing." - Archilochus
\item Hedgehogs - Big ideas that govern all motion around us. Think Karl
Marx
\item Foxes - little ideas explain a little of an issue, but many of them
combine to explain the picture. Usually multidisciplinary, cautious
and empirical, as well as tolerant of complexity.
\item Hedgehogs however make much better television guests, and receive
the lion's share of publicity despite being worse at predictions.
\end{itemize}
\end{frame}

\begin{frame}[label=sec-2-2-4]{Lessons from Hedgehogs}
\begin{itemize}
\item Hedgehogs get worse at predicting as they get more experience
\item They are letting their biases take over their prediction
\item This occurs with television pundits and partisanship.
\item Don't construct narratives, these become less like reality the more
information you get, and become worse predictors.
\end{itemize}
\end{frame}

\begin{frame}[label=sec-2-2-5]{How to Apply this to Prediction}
\begin{itemize}
\item Think Probabilistically ( Bayesian v Frequentist )
\item Keep making forecasts and don't let the past affect today's prediction
\item Consensus is a good sign, but not always
\item There is no single data point that is sufficient for a complex
system.
\item Weigh Quantitative and Qualitative data together - Cook Political
Report.
\end{itemize}
\end{frame}

\subsection{Baseball}
\label{sec-2-3}

\begin{frame}[label=sec-2-3-1]{What makes baseball so easy to predict}
\begin{itemize}
\item There is a lot of good data
\item Play is largely independent. Players responsible for their own
statistics
\item Relatively few problems with complexity or dynamics
\item We understand the processes underlying the data
\item Easy to account for unobserved heterogeneity.
\end{itemize}
\end{frame}

\begin{frame}[label=sec-2-3-2]{PECOTA isn't God's gift to scouting}
\begin{itemize}
\item Outclassed by Professional scouting in predicting players' ability
from minor to major league.
\item A hybrid approach is the best predictive measure.
\item Especially since there is interdependence between certain aspects
(Pitchers in minor leagues aren't playing against great hitters).
\item Bias can happen in statistical models - Billy Beane had bad
defenses.
\end{itemize}
\end{frame}

\begin{frame}[label=sec-2-3-3]{Don't Limit the Data}
\begin{itemize}
\item Being able to combine qualitative information is important for
prediction
\item Baseball scouts exist to find out this information and bring to
attention
\item Beware of discounting something we can't categorize easily such as
the tool kit of Dustin Pedroia
\end{itemize}
\end{frame}

\subsection{Weather Prediction - A Success Story}
\label{sec-2-4}

\begin{frame}[label=sec-2-4-1]{Predicting Weather}
\begin{itemize}
\item Laplace's Demon - If you knew the exact state of the universe and
all the laws of motion, we could predict perfectly.
\item While this may be false, if we know the laws of motion and know the
states of the universe, we're going to make pretty good predictions.
\item Weather is something that we understand the underlying motion very well.
\item We don't however know the state of every molecule in Earth's atmosphere
\end{itemize}
\end{frame}

\begin{frame}[label=sec-2-4-2]{Predicting Weather}
\begin{itemize}
\item We can however, discretize space and apply equations to data
gathered. First attempted by Lewis Richardson
\item However there is a lot of number crunching, so computers are pretty
much required.
\item However, as you become more precise in the discretization of the
atmosphere, there is a computational cost. Doubling Precision
increases the grid by 16 times as much.
\item However this is a dynamic system, with one dimension being time, and
we are not very good at predicting these.
\end{itemize}
\end{frame}

\begin{frame}[label=sec-2-4-3]{Chaos Theory}
\begin{itemize}
\item Applies to non-linear dynamic systems.
\item Tiny changes in the initial conditions can have huge consequences in
the outcomes, even in a completely deterministic model.
\item This means measurement error, and human error can have huge
consequences in the predictions.
\item This is handled by perturbation of the initial conditions and simulation.
\end{itemize}
\end{frame}

\begin{frame}[label=sec-2-4-4]{The Human Touch}
\begin{itemize}
\item It doesn't end there, there is still a human element to weather predictions.
\item Humans attempt to account for mistakes that the computer may make,
through their intuition about the machinations of the model.
\item This is true because humans are much better still at visually
identifying patterns and can correct for it much better than a
computer can.
\item This can lead to problems such as over fitting, but meteorologists
improve predictions by about 25\%, and predictions keep improving
\end{itemize}
\end{frame}

\begin{frame}[label=sec-2-4-5]{Evaluating a Forecast}
\begin{itemize}
\item Industry competition is not always seeking a more precise forecast,
but a more useful one.
\item The Economist's view on the Receiver Operating Curve.
\item This can lead to dishonest predictions that are not necessarily bad,
they simply are responding to other objectives.
\item Be aware of goals other than strictly reporting truth.
\end{itemize}
\end{frame}


\section{Prediction - Where it really doesn't work}
\label{sec-3}
\subsection{Earthquakes}
\label{sec-3-1}

\begin{frame}[label=sec-3-1-1]{Not such a success story - Earthquakes}
\begin{itemize}
\item Officially "cannot be predicted" - Only Forecasted
\item While we know the long-term rates at which earthquakes hit and the
probabilities of earthquakes of different intensities, we can't
really tell anything in the short-term.
\item No real understanding of the mechanics that are working "under the
hood"
\item Data is relatively poor on anything but relatively powerful quakes
in many places.
\end{itemize}
\end{frame}

\begin{frame}[label=sec-3-1-2]{Purely Statistical Models Don't Work}
\begin{itemize}
\item While aftershocks may be able to be predicted, especially involving
movement down fault lines, we would like to predict the first shock.
\item Without theory, you will simply be finding patterns in the noise.
\item It is impossible to identify the signal in data that is this noisy
without knowing what it is you are looking for.
\item This is different from weather prediction, since we understand the
laws of motion there.
\item It is too easy to over fit because of the noise and large amount of data.
\end{itemize}
\end{frame}

\begin{frame}[label=sec-3-1-3]{Why they aren't working (and may never)}
\begin{itemize}
\item The Data is extremely noisy.
\item "With four parameters I can fit an elephant, with five I can make
him wiggle his trunk" - Von Veuman
\item Making a model more complex often makes it worse.
\item Non parametric fits that diverge from theory can be from over fitting
and missing rare events
\item Complex Process - Many simple actors combine for a very complicated process
\end{itemize}
\end{frame}

\subsection{Economics}
\label{sec-3-2}

\begin{frame}[label=sec-3-2-1]{Economics - Not so good with the predictions}
\begin{itemize}
\item Economists are very bad at predicting, and often don't the
uncertainty of their forecasts
\item Often too overconfident - Despite there being readily available
feedback
\item Predictions fail to be accurate or precise
\item "Nobody has a clue" - Jan Hatzius
\end{itemize}
\end{frame}

\begin{frame}[label=sec-3-2-2]{Why do the predictions fail so badly}
\begin{itemize}
\item The underlying processes are both dynamic and complex
\item Cause and effect are extremely hard to determine
\item The Data is horrible
\item The Lucas Critique
\item Goodhart's law - Targetting an indicator removes its predictive
power
\end{itemize}
\end{frame}

\begin{frame}[label=sec-3-2-3]{Structural Changes}
\begin{itemize}
\item Since the economy is a complex dynamic system, it is changing over
time.
\item In particular becoming a global economy has shifted capital levels
\item Baby boomers retiring, declining middle class and increased debt
have all altered the way the American Economy works.
\item Much of the recent growth during The Great Moderation was
debt-fueled
\end{itemize}
\end{frame}

\begin{frame}[label=sec-3-2-4]{Can these problems be handled?}
\begin{itemize}
\item Don't throw out data - Silver claims the fed overvalued data from
the Great Moderation to determine their forecasts
\item To not throw out data is to assume there is no structural change
over time, must make an assumption one way or another.
\item How do we know when the economy has changed structurally?
\end{itemize}
\end{frame}

\begin{frame}[label=sec-3-2-5]{Why have many of the Problems been resolved in weather and not economics?}
\begin{itemize}
\item There is no hard science to turn back to.
\item The structural approach which attempts this still has to make many
unrealistic assumptions
\item The system is far more complex
\item Agents are all individually making decisions.
\end{itemize}
\end{frame}

\begin{frame}[label=sec-3-2-6]{Computers don't help}
\begin{itemize}
\item They just give you faster and bigger ways to mistake the signal for
the noise.
\item Models need to be about the economics going on behind the scenes,
not the data that is produced by the economics.
\item Stories about data are still just made by hedgehogs, and are
unlikely to be anything but over fitting noise.
\end{itemize}
\end{frame}

\begin{frame}[label=sec-3-2-7]{Incentives}
\begin{itemize}
\item Forecasters who do not have their names attached to forecasts
usually do better
\item They aren't trying to make a name for themselves or trying to stay
conservative to damage established reputation.
\item Market for predictions or changing how we consume forecasts?
\end{itemize}
\end{frame}

\subsection{The Flu}
\label{sec-3-3}
\begin{frame}[label=sec-3-3-1]{Predictions can influence behavior}
\begin{itemize}
\item Predictions can undermine themselves - GPS diverts all traffic and
creates a bigger jam
\item Economic Predictors of recessions make actually cause them
\item Flu predictions can increase vaccinations and reduce flu damage
\end{itemize}
\end{frame}

\begin{frame}[label=sec-3-3-2]{A Solution to Complex Systems?}
\begin{itemize}
\item When systems are complex because of actors - they can be simulated
\item However these simulations need all the actions of the actors
predicted, often by models that are unreliable.
\item Worse they can come from psychology, or just be made up by the
creators.
\item For now have relatively little predictive power and are very hard to
test
\end{itemize}
\end{frame}

\section{Predicting Better}
\label{sec-4}
\subsection{Thinking Bayesian}
\label{sec-4-1}
\begin{frame}[label=sec-4-1-1]{False Positives}
\begin{itemize}
\item Bayes theorem tells us that if the underlying probability of
something is very low then false positives can dominate the results.
\item Tests that are 95\% accurate for rare diseases can only be 10-30\%
sure that you have the disease.
\item Research is especially affected by this because of the bias for
results and the "publish or die" mentality.
\end{itemize}
\end{frame}

\begin{frame}[label=sec-4-1-2]{Fisherian vs Bayesians}
\begin{itemize}
\item Frequentists take a strong stance on the structure of the problem
\item This eliminates the prior, but also eliminates sources of error
besides the sampling error
\item Requires a sample and many other assumptions that give it
computational tractability at the cost of rigidity.
\item Since the prior is eliminated it eliminates context which takes the
form of a Bayesian Prior.
\end{itemize}
\end{frame}

\begin{frame}[label=sec-4-1-3]{Gambling}
\begin{itemize}
\item Bob Voulgaris - Bayesian approach to betting - everything is looked
at in the context of probabilities.
\item No notion of statistical significance, so there is no arbitrary
cutoff levels at 95\%.
\item When rewards are occurring at the margin, as is the case in almost
all competitive predictive markets, this is the approach that is
needed.
\item Hypothesis are still tested and examined in the context of some
theory, but not tested in the Null hypothesis framework.
\end{itemize}
\end{frame}

\begin{frame}[label=sec-4-1-4]{Less structure means more work}
\begin{itemize}
\item Since there is much less structure placed on the model, our beliefs
weigh stronger into the prediction.
\item But no matter what our prior beliefs are, with enough data we should
still see our beliefs converge.
\item The literature is also on the path towards Bayesian approaches and
is "converging" there.
\end{itemize}
\end{frame}

\subsection{Computers - Where they are useful and where they are not}
\label{sec-4-2}

\begin{frame}[label=sec-4-2-1]{Understand the underlying processes}
\begin{itemize}
\item Computers have drastically helped weather prediction, where we
understand the fluid dynamics that explain their machinations.
\item Relatively useless at predicting things in Economics and Earthquakes
where we don't understand the underlying process.
\end{itemize}
\end{frame}

\begin{frame}[label=sec-4-2-2]{Computers work better with good data}
\begin{itemize}
\item Computers are very likely to mistake the signal for the noise
\item They lack the sight to see the big picture, and compensate by brute
force and look-ahead algorithms with pruning.
\item However they are still victims of the assumptions that their
programmers have put on them.
\end{itemize}
\end{frame}

\subsection{Poker}
\label{sec-4-3}

\begin{frame}[label=sec-4-3-1]{Poker}
\begin{itemize}
\item Pareto Principle - Diminishing returns to skill in poker
\item Very noisy output - randomness plays a large role in returns
\item Feeding on overconfidence of the fish.
\end{itemize}
\end{frame}

\begin{frame}[label=sec-4-3-2]{Tilting}
\begin{itemize}
\item Tilting results from the fact that skill matters, but the game is
still extremely noisy, and this can effect how we play.
\item In all things avoid entitlement tilting - thinking that because you
acted optimally, that you deserved to win.
\item Don't be seduced by noisy positive results meaning that you are
doing better
\item Cannot evaluate predictions in noisy areas solely by right and
wrong, have to evaluate the process.
\end{itemize}
\end{frame}

\subsection{Group Forecasts}
\label{sec-4-4}

\begin{frame}[label=sec-4-4-1]{The Wisdom of Crowds}
\begin{itemize}
\item Aggregation is incredibly powerful in certain situations
\item However when people are reacting to other bets made, as is the case
in a market, the dynamics are not as simple.
\item The aggregate is only sometimes better than the best individual prediction.
\end{itemize}
\end{frame}


\section{Vaguely Political Things}
\label{sec-5}

\subsection{Climate Science}
\label{sec-5-1}

\begin{frame}[label=sec-5-1-1]{Climate Prediction}
\begin{itemize}
\item We understand the fundamental process that is heating the earth - The Greenhouse effect.
\item However the primary cause for heating the earth isn't C0$_{\text{2}}$, its the
water vapor stored in the air that becomes possible when the earth
heats.
\item While we understand the root cause very well, we are not very good
at predicting temperatures.
\end{itemize}
\end{frame}

\begin{frame}[label=sec-5-1-2]{Problems}
\begin{itemize}
\item Fancy models in the time from 1995-2012 were outperformed by a
linear regression on C0$_{\text{2}}$ only.
\item Complex systems that are influenced by lots of noise and cyclical
trends like El Niño rarely perform well.
\item Scott Armstrong bet Al Gore in 2007 that temperatures would not
increase each month over the next Decade and won.
\item Large structural uncertainty and initial condition uncertainty.
\item From 2001 to 2011 temperatures declined (though not by much)
\end{itemize}
\end{frame}

\begin{frame}[label=sec-5-1-3]{A Bayessian Explanation}
\begin{itemize}
\item Based on some prior belief of temperatures increasing, we can look
at how our beliefs in global warming can change with a decade of
temperatures decreasing.
\item The more certain you are in your prior, the more hurt you are
going to get by being wrong.
\item 95\% sure of Global warming updates to 85\%
\item 99\% sure of Global warming updates to 28\%.
\item Overconfidence in something can reduce public faith when you are
incorrect.
\end{itemize}
\end{frame}

\subsection{Unknown Unknowns}
\label{sec-5-2}

\begin{frame}[label=sec-5-2-1]{Uncertainty is back}
\begin{itemize}
\item When we are unfamiliar with a possibility, we don't even think about
it happening.
\item "there are also unknown unknowns \ldots{} things we do not know we don't
know" - Donald Rumsfeld
\item It can be very hard to sort and build theories of what we should be
looking for when there isn't a solid theory to lean back on
\item Failures of predicting 9/11 and Pearl Harbor.
\end{itemize}
\end{frame}

\section{Conclusions}
\label{sec-6}

\subsection{Conclusion}
\label{sec-6-1}

\begin{frame}[label=sec-6-1-1]{In Summary}
\begin{itemize}
\item Try to understand the theory underlying a process
\item Think probabilistically and Bayesian
\item Stay simple, but don't make so many assumptions that you lose
predictive power
\item Keep your incentives on getting predictions right, and do not trust
those without the same incentives.
\end{itemize}
\end{frame}
% Emacs 25.1.1 (Org mode 8.2.10)
\end{document}