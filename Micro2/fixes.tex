% Created 2018-04-03 Tue 13:42
\documentclass[11pt]{article}
\usepackage[utf8]{inputenc}
\usepackage[T1]{fontenc}
\usepackage{fixltx2e}
\usepackage{graphicx}
\usepackage{longtable}
\usepackage{float}
\usepackage{wrapfig}
\usepackage{rotating}
\usepackage[normalem]{ulem}
\usepackage{amsmath}
\usepackage{textcomp}
\usepackage{marvosym}
\usepackage{wasysym}
\usepackage{amssymb}
\usepackage{hyperref}
\tolerance=1000
\usepackage{natbib}
\author{Timothy Schwieg}
\date{\today}
\title{Quiz 4 Corrections}
\hypersetup{
  pdfkeywords={},
  pdfsubject={},
  pdfcreator={Emacs 25.1.1 (Org mode 8.2.10)}}
\begin{document}

\maketitle

\section{Question 1b.}
\label{sec-1}

\subsection{Pooling Equilibrium:}
\label{sec-1-1}
The only pooling equilibiria that can form is when nobody buys any
insurance, or insurance is offered at such a low price that everyone
purchases insurance. For any other pricing strategy by the firm, some
consumers will decide not to buy the insurance, and this will devolve
into a semi-seperating equilibrium. 

\subsubsection{No Insurance is bought}
\label{sec-1-1-1}
If no insurance is bought, the profit is zero.

\subsubsection{Everyone buys insurance}
\label{sec-1-1-2}
This requires that every person is willing to buy insurance at the
price of p', this means that h(p') = 0, as for any price below that,
the firm would have an incentive to raise prices. Therefore profit is
given by:
$$p - L \mathbb{E}[ \pi ] = p - L \int_0^1 \pi dF(\pi)$$


\subsection{Semi-seperating Equilibrium:}
\label{sec-1-2}
In a semi-seperating equilibrium, the consumers seperate themselves by
either purchasing insurance, indicating that their type is above some
threshold, or they do not purchase insurance. This means that the
insurance company only pays out, and recieves premiums from those they
are purchasing insurance, and their profit is given by: 

$$p - L \mathbb{E}[ \pi_i | \pi_i \geq h(p' ) ] = p - \frac{ L \int_{h(p')}^1 \pi
dF(\pi)}{1 - F(p')}$$


\section{Question 4}
\label{sec-2}
\subsection{Part a}
\label{sec-2-1}
For the L type to not seek tenure, we must require that his payoff is
lower from the behavior seeking tenure than from his payoff for
planning to attend Boondocks. Therefore we require that $2000 - 60N \leq
500$. This leads to $60N \geq 1500 \text{ and } N \geq 25$, thus we
     must have N be at least 25 in order to prevent the L type from
     seeking tenure.

\subsection{Part b}
\label{sec-2-2}
For H to still be willing to participate, we need his utility from
attempting to obtain tenure to be higher than working at Boondocks, so
we require that: $2000 - 30N \leq 500$, this reduces to $1500 \leq 30N$, and $N
\geq 50$. 50 is therefore the highest number of publication that can be
required to still have H types willing to participate.
% Emacs 25.1.1 (Org mode 8.2.10)
\end{document}