\documentclass[10pt]{paper}

\usepackage{amsmath}
\usepackage{float}
\usepackage{amssymb}
\usepackage{geometry}

\title{Quiz 2}
\author{Timothy Schwieg}

\begin{document}

\maketitle

\section*{7.5}

\subsection*{a}

This claim is false, consider the case when $N = 2$, then the pure
strategy of picking one strictly dominates the pure strategy of
picking 100.

\begin{tabular*}{1.0\linewidth}{l|l|l|l}
  & 1 & 2 - 99 & 100\\\hline
  Pure 100 & 0 & 0 & 50\\
  Pure 1 & 50 & 100 & 100\\
\end{tabular*}


As we can see, regardless of the move made by the other player in the
2 person game, picking 1 always fares better than picking 100, and
there is a pure strategy that strictly dominates picking 100, at least
in the case of 2 players.

\subsection*{b}

Consider the strategy of picking any number with equal
probability. Since the payoff of picking 100 is 0 for all cases except
when everyone picks 100, we need only show that this strategy has a
payoff greater than zero. Consider the case when not every player has
picked 100. There is a number that is closest to the average divided
by 3, and we have a positive probability of picking that number, so we
have an expected utility that is greater than zero, and we are
strictly better off than picking 100. 

In the case when every other player picks 100, then with probability
$\frac{1}{100}$ we pick 100 and earn 50\$, but with probability
$\frac{99}{100}$ we earn 100\$ and are strictly better off than
picking 100. Thus this strategy strictly dominates the pure strategy
of picking 100.

\subsection*{c}

Consider the case when everyone else picks 100, then the payoff from
picking 99 is 100\$. This is the maximum amount that a person could
earn from the game, and can therefore not be strictly dominated.

\subsection*{d}

Since we can see that 100 is strictly dominated, we may iteratively
eliminate this strategy. Our new strategy set for all palyers is
picking betweeen 1-99. But the rules are the same in this game, so we
should follow the same logic and eliminate the pure strategy of
choosing 99. This leaves us after 2 rounds with the choices 1-98. This
will continue by eliminating the highest numbered choice as it is
strictly dominated.

After 99 iterations, the only choice left is 1 and there is a unique
choice left over.

\subsection*{e}

Let us approach this by fixing the choice of player 3 at 100. Consider
three arbitrary cases: Player two picks from 15-100, Player two picks
14, and Player two picks 1-13.

\subsubsection*{15-100}

If player two picks between 15-100, then the smallest that the average
divided by three must be where his guess is 15: $\frac{115+x}{9}$.
By choosing x so that we are closest to this number, we choose x =
14. For any number greater than 15, if player 1 chooses 14, we will
still obtain that 14 is the closest number. This occurs since:
$$\left | \frac{114+y}{9} - y \right | \geq \left | \frac{114+y}{9} - 14
\right | \quad \forall y \in [15,100] \subset \mathbb{Z}$$
So for player 3 fixed at 100, and player 2 choosing 15-100, player 1
finds that 14 weakly dominates every other number.

\subsubsection*{14}
If player two picks 14, and player 3 picks 100, then the best strategy
for player 1 is to pick 14 as well.

\subsubsection*{1-13}
Since we can see that $\left | \frac{ 114+y}{9} - y \right| \geq \left |
  \frac{114+y}{9} - 14 \right | \forall y \in [1,13] \subset \mathbb{Z}$.

\begin{tabular*}{1.0\linewidth}{l|l|l}
y & $\left | \frac{114+y}{9} - 14 \right |$ & $ \left | \frac{114+y}{9}- y \right |$\\\hline
y = 1 & 2 & 11\\
y = 2 & 2 & 10\\
y = 3 & 1 & 10\\
y = 4 & 1 & 9\\
y = 5 & 1 & 8\\
y = 6 & 1 & 7\\
y = 7 & 1 & 6\\
y = 8 & 1 & 5\\
y = 9 & 1 & 4\\
y = 10 & 1 & 3\\
y = 11 & 1 & 2\\
y = 12 & 0 & 2\\
\end{tabular*}

Clearly choosing 14 weakly dominates all other choices for player 3
fixed at 100. 

From this we can see that 14 weakly dominates all other choices when
player 3 picks 100. We can lower player 3's guess, and without loss of
generality state that player 2 can only pick between 0-99. From here
we can see that the average will only be lowered, so there will never
be a strategy higher than 14 that weakly dominates the
others. As the max number lowers by 3, the weakly dominant strategy
will be lowered, so we will see that each number from 1-14 will weakly
dominate all other numbers for some choices of player 2 and 3. Since
there is no time when 15-100 will weakly dominate all other choices,
we find that those numbers are weakly dominated in the first step, and
1-14 are the set of weakly dominant strategies.

We may repeat this procedure, reaching only the options of \{1,2\}
after the second iteration, and \{1\} after the third iteration.

\section*{Question 3 }

We can see plainly that both players want the combined effort level to
be at 10. If they were willing to put forth effort, and the combined
effort was greater than 10, they would benefit by using less effort
until the combined level was 10. They would then be strictly better
off since they are recieving the same benefit and excerting less effort.

If the effort level was less than 10, they could increase their effort
level such that the combined effort level was 10, excerting effort
less than or equal to 10, and reaping the benefit of 100\$, making
them strictly better off.

Thus any combination of effort such that they both combine to excert
10 effort is a Nash equilibrium. Without loss of generality, fix the
effort of player 2, the best response of player one is to not change
his effort, since lowering it costs him 100\$ and at best saves him
10\$ in effort. Raising it gains him nothing, and costs him however
much he raises his effort by. Thus this is a Nash equilibrium. 

\end{document}