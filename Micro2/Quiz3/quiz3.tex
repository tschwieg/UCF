\documentclass[10pt]{paper}

\usepackage{amsmath}
\usepackage{float}
\usepackage{amssymb}
\usepackage{geometry}

\title{Quiz 3 Corrections}
\author{Timothy Schwieg}

\begin{document}
\section*{1. Nash Equilibrium of an Infinitely repeated game}

Consider the trigger strategy of each player playing No, unless the
opponent plays yes, in which case in all future repetitions the player
plays yes.

Each player discounts the future based upon the interest rate and the
probability that the game will continue into the next period. The
probability that the game will continue is given by $p = 1 - \sqrt{ .9 }$ and
the interest rate is: $r = \frac{1}{\sqrt{.9}} - 1$. Therefore the
discount factor is given by $\delta = \frac{p}{1+r} = \frac{1 - \sqrt{.9}}{1 +
  \frac{1}{\sqrt{.9}}-1} = \sqrt{ .9 } - .9$

The return to a player for playing No at a time period is $R_{No} = 4
\sum_{t=0}^\infty \delta^t = \frac{4}{1-\delta} = \frac{4}{1.9 - \sqrt{ .9 }} \approx 4.205$.

The return for deviating at any time period is given by: $R_{Yes} = 7
+ \sum_{t=1}^\infty \delta^t = 7 - 1 + \sum_{t=0}^\infty \delta^t = 7 - 1 + \frac{1}{1-\delta} = 6 +
\frac{1}{1.9 - \sqrt{.9}} \approx 7.051$

We can see that it is beneficial for players to deviate from the
strategy of playing No each round, so there does not exist a Nash
equilibrium where each player plays No in all periods.

\section*{2. Folk Theorem}

Let the dimension V of the space of feasible payoffs be equal to the
number of players, then for any $v \in V$ with $v_i > v_{-i} \quad \forall$
players i, $\exists \underline{\delta} < 1$ such that $\forall \delta \in ( \underline{\delta},1) \quad
\exists$ a subgame perfect Nash Equilibrium of $G(\delta)$ with payoffs v.

Feasible means that each player is doing better than their minmax
payoffs, and dimension of a vector space is minimum number of linearly
independent vectors required to span the space. 

Proof sketch: All players are playing according to some strategy
without deviation. If player i deviates from said strategy, all other
players minmax him for N periods with N and $\delta$ chosen such that no
other player wishes to deviate from this strategy.

If nobody deviates from this minmax strategy, after N periods all
other players are awarded $\epsilon$ above the minmax of i, and i continues
to get his minmax. However, if any player deviates from the strategy,
he is minmaxed by the same rules applied to player i.

This reward after minmaxing incentivises players to minmax the
deviants, while keeping the original deviant in a bad situation. 

\section*{4. Subgame Perfect Nash Equilibrium}

\subsection*{a}

Note that if Firm E has decided to choose In, he finds that
accommodate strictly dominates fight.

This reduces firm I's information set, eliminating the option of Firm
E choosing fight. Since Firm I knows that he is at In, Accommodate, he
chooses to accommodate.

Since Firm E knows that Firm I will choose accommodate, he knows that
if he chooses in, he will recieve a return of 3, which is higher than
the return of 0 for being Out. This means that he will choose In. 

This leaves the strategies of Firm E choosiong: In, Accommodate.
Firm I chooses: If In, choose accommodate, If Out, No choice. 

\end{document}