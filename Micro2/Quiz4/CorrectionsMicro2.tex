% Created 2018-04-17 Tue 13:01
\documentclass[11pt]{article}
\usepackage[utf8]{inputenc}
\usepackage[T1]{fontenc}
\usepackage{fixltx2e}
\usepackage{graphicx}
\usepackage{longtable}
\usepackage{float}
\usepackage{wrapfig}
\usepackage{rotating}
\usepackage[normalem]{ulem}
\usepackage{amsmath}
\usepackage{textcomp}
\usepackage{marvosym}
\usepackage{wasysym}
\usepackage{amssymb}
\usepackage{hyperref}
\tolerance=1000
\usepackage{minted}
\author{Timothy Schwieg}
\date{\today}
\title{Micro Quiz Corrections}
\hypersetup{
  pdfkeywords={},
  pdfsubject={},
  pdfcreator={Emacs 25.3.1 (Org mode 8.2.10)}}
\begin{document}

\maketitle

\section{Question 2}
\label{sec-1}
\subsection{Part b}
\label{sec-1-1}
The insurance company wishes to ensure that the consumer will still
participate, so his utility under e = 1 must be higher than not having
insurance.

His utility for having insurance is given by: $\frac{2}{3} \sqrt{
100 - p } + \frac{1}{3} \sqrt{ 100 - p - L + B } - \frac{1}{3}$ and his
utility if he does not purchase insurance is: $\frac{1}{3} \sqrt{
100 - L } + \frac{2}{3} \sqrt{ 100 } - \frac{1}{3}$

His condition for ensuring participation for the consumer must then
be:
\begin{align*}
\frac{2}{3} \sqrt{ 100 - p } - \frac{1}{3}\sqrt{ 100 - p - L + B }
&\geq \frac{1}{3} \sqrt{ 100 - L } + 
\frac{2}{3} \sqrt{ 100 } \\
\sqrt{ 100 - p } &\geq \frac{1}{3} \sqrt{49} + \frac{2}{3} \sqrt{ 100
}\\
\sqrt{ 100 - p } &\geq \frac{7}{3} + \frac{ 20}{3} \\
( 100 - p ) &\geq 81\\
p &\leq 19
\end{align*}

\section{Question 3}
\label{sec-2}
The consumer receives marginal benefit of $P(n-1) - P(n)$ with each
additional search, and faces marginal cost of c. The consumer wishes
to continue searching as long as his marginal benefit is greater than
his marginal cost. So he will choose to search 
\begin{align*}
n^* = \max_{n \in
\mathbb{Z}^+} n \text{ such that: } P(n-1) - P(n) \leq c
\end{align*}

To prove that this search rule is in fact optimal, first we will
establish monotonicity then approach via Contradiction. 

Monotonicity: We wish to show that $P(n-1) - P(n)$ is an decreasing
function of n. 

\begin{align*}
P(n) = K \int_0^1( 1- F(p))^n dp
\end{align*}

Let $L(n) = P(n-1) - P(n)$. We wish to show that $L(n)$ is an
decreasing function:

\begin{align*}
L(n+1) - L(n) = 
P(n) - P(n+1) - P(n-1) + P(n) = \\
K \left ( \int_0^1 (1-F(p))^n - (1-F(p))^{n+1} - (1-F(p))^{n-1} + (1-F(p))^n dp \right )\\
K \left ( \int_0^1 (1-F(p))^{n-1} \big ( (1-F(p)) - (1-F(p))^2 - 1 + (1-F(p)) \big ) dp
\right )\\
K \left ( \int_0^1 (1-F(p))^{n-1} \big ( 1 - F(p) - 1 + 2F(p) - F(p)^2 - 1 + 1 - F(p)
\big ) dp \right )\\
K \left ( \int_0^1 (1-F(p))^{n-1} ( -F(p)^2 ) dp \right ) < 0\\
\end{align*}


Assume that $n^*$ is not optimal. Let $n' \neq n^*$ be the any optimal
number of searches to make that minimizes the costs faced by the
agent. Either $n' < n^*$ or $n' > n^*$.

Case: $n' < n^*$ Consider the costs faced by searching $n' + 1$
times. The difference of costs between this and n' is given by: $(n' +
1)c + P(n' + 1 ) - n' c - P(n' ) = c + P(n' + 1) - P(n')$. Since it is
known that $n' < n^*$, $P(n' ) - P(n' + 1) \leq c$ so $c + P(n' + 1) -
P(n') \leq 0$. This contradicts $n'$ being the minimum of costs, as $n' +
1$ has lower costs.

Case: $n' > n^*$ .  Since n$^{\text{*}}$ is the maximum n such that $P(n-1) - P(n)
\leq c$, it must be true that $P(n'-1) - P(n') > c$.  Consider the
difference between the cost of n' and n' - 1 given by: $(n' -1)c - n'
c + P(n' - 1) - P(n')$. This is positive since it reduces to have
$P(n'-1) - P(n') - c$ Since it is positive, $n'$ cannot be the minimum
of the costs, as the cost of $n' - 1$ is less.  Therefore it is not
possible for $n' > n^*$.

This is a contradiction to $n' \neq n^*$, so there cannot be a minimum to
costs that is not equal to $n^*$
% Emacs 25.3.1 (Org mode 8.2.10)
\end{document}