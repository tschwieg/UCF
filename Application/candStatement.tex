% Created 2018-04-19 Thu 11:16
\documentclass[10pt, a4paper]{article}
\usepackage[utf8]{inputenc}
\usepackage[T1]{fontenc}
\usepackage{fixltx2e}
\usepackage{graphicx}
\usepackage{longtable}
\usepackage{float}
\usepackage{wrapfig}
\usepackage{rotating}
\usepackage[normalem]{ulem}
\usepackage{amsmath}
\usepackage{textcomp}
\usepackage{marvosym}
\usepackage{wasysym}
\usepackage{amssymb}
\usepackage{hyperref}
\tolerance=1000
\usepackage{minted}
\usepackage[margin=1in]{geometry}
\pagestyle{headings}
\markright{Timothy Schwieg}
\author{Timothy Schwieg}
\date{}
\title{Candidate Statement}
\hypersetup{
  pdfkeywords={},
  pdfsubject={},
  pdfcreator={Emacs 25.3.1 (Org mode 8.2.10)}}
\begin{document}

\maketitle
Hello, my name is Timothy Schwieg and I am a student at the University
of Central Florida. I am currently earning my masters degree in a
Business analytics program, and this is not what I imagined I would be
doing at this point in my life. I went to a competitive high school
and I did fairly well. I was accepted to Georgia Tech with some
aid. However, It turned out to not be enough, and I found myself
unable to pay for my out of state tuition there. Frustrated and angry
I enrolled at UCF because it was close and I knew I could get help
even if I was applying as extremely late as I was.

For a while I was too angry at my situation to be a good student, and
I earned a few B's in classes I really shouldn't have. Worse yet, I
was forming all the bad habits that come with apathy and
resentment. However I was doing well enough that this wasn't really
affecting me, and it wasn't until I decided to continue into graduate
school that I realized what I had become. Sure I had the occasional B,
or a few A-, but it wasn't really crushing my GPA. I had fallen into a
slump where I accepted mediocrity. Like all slumps, it takes a while
for you to realize its magnitude, and a shock to take you out of it.

The shock that woke me up took the form of Dr. Harry Paarsch teaching
Econometrics. He had a whole routine to scare the poor
business students, but at the core of it he was searching for the
students willing to signal their ability to put their heads down and
finish something they started. That push was exactly what I needed. He
had assigned a term paper for the class, and I launched myself into
it, working with a professor in the math department, and trying to
build my own model of poker play. It didn't go very well in terms of
predictive power, but that has never stopped any economist. I don't
think the model I built was even close to something worth putting
production. But it was never meant to be anything more than an
exercise in creation.

Something changed working on that paper. I realized this was what I
wanted to do. I knew I needed to know more so that I could begin
applying serious models. That drove me to finally understand that
graduate school was where I needed to go. While I had been teaching, and I enjoyed it,
it was the feeling of putting together research that really connected
with me. I found the purpose I had been missing at UCF before, and it
galvanized my desire. I knew I had to do whatever I could to put
myself in a position to be a good researcher. It wouldn't be enough to
do well at UCF, I needed to be more.

After the semester had finished, I spoke with Dr. Paarsch about
graduate school, and working on a paper to submit as a writing
sample. He then did something no other professor at UCF had ever done.
He didn't try to tell me I could go to mystical far-away places; he
told me I would come short. He told me that I needed to take time
learning how to be a serious student, how to push myself and keep
myself disciplined. He was absolutely correct. So I enrolled in the
economics/business analytics program and set myself up to fix these
problems. I believe that in the last year I have succeeded in that. I
have not only devoted myself to working hard, but also pushing myself
to trying new problems.

The program I am in has lent itself to exploring these new problems,
especially in computing. We have a lot of freedom in how deep we wish
to explore them. Dr. Paarsch enjoys asking very open-ended questions
such as: How much should you save for a house. This gives us a lot of
choice on the complexity of the model we want to use, from a simple
percent of income to a stochastic dynamic program. While most of my
attempts at implementing models such as an LQ-control problem for
savings weren't perfect, exploring the dynamics was an experience. In
every open-ended problem I have explored and simulated the problem before
trying to answer it. 

I failed on some of them. Those problems were the ones I learned the
most from. Breaking out of my comfort zone was more than just hard,
but it has taught me exponentially more than keeping myself
constrained. Maybe I broke a few more eggs than I made omelets, but I
made more omelets than ever before. I want to continue to expose
myself to new ideas and strategies, continue to push myself, and learn
while preparing myself for a doctoral program. I know I'll fail,
probably many times, but I also know I'm going to keep trying until I
finally succeed.

A year ago I wasn't sure exactly what I wanted to do. However, the
more time I've spent in economics, the more certain I've been that
this is what I want to do with the rest of my life. I've prepared
myself for the mathematical rigor I will face, and I am ready to take
the plunge and prove to the world that I can be at the top, and I
deserve to be there. I've come from UCF, but I won't let that define
what I've become.

I am primarily interested in Econometrics, specifically the structural
approach. Much of my experience has been in auctions where a
structural approach is very powerful. I believe that where the theory
has been proven the structural approach is the best lens through which
to interpret the data. Viewing a model through primitives that are
then estimated is in my opinion, the best way to proceed. My main
introduction to this material has been through \emph{The Structural
Econometrics of Auctions} by Paarsch. Throughout the first seven weeks
in the Seminar in Economic Topics in my program we worked through many
sections in the book, such as in what contexts can valuation
distributions be identified.

One exception to the structural approach that I have found to be very
interesting is many behavioral economic models. It can be difficult or
impossible to fit some behavioral models. For example, I tried (and
failed) to incorporate an auction model into a model of valuations
(cumulative prospect theory) that required values be interpreted as
relative to the price of the lottery. It did not go well, and a more
reduced form approach was required. I am very interested in the
approaches required to bring these two together, and see how the
structure of the behavioral model might change the behavior suggested
by the theory. I believe that the combination of these two will be
where the best predictive power in economics will come.

During my undergraduate I focused on Probability and stochastic
processes. The course I took titled \emph{Topics in Applied Mathematics}
was a course on financial mathematics, and the course Stochastic
Processes continued on this trend. I have also completed some
self-readings in analysis beyond the work I did for \emph{Advanced
Calculus}, completing material out of Rudin's \emph{Principles of
Mathematical Analysis}. I believe that this makes up for the fact that
I only took a single semester of analysis officially at the
university. 

One area that I have worked in and found very interesting was
numerical methods in dynamic programming. While I have not worked on
very complicated dynamic programs, I worked on several assignments for
Dr. Paarsch where I implemented dynamic programs numerically. I
implemented both the Ramsey problem for a set consumption function
and a variety of stochastic shocks affecting the production function
in two manners: one in python using existing numpy libraries and
vectorized code and in Julia where all the methods were implemented by
myself. The techniques used were various forms of Gaussian quadrature
for numerical integration, fitting a Chebyshev polynomial for the
value function, and iteratively applying the contraction to find the
equilibrium fixed point. Another model that I implemented was an
explore-exploit model of a two-armed bandit that featured Bayesian
updating to determine the optimal choice.

Much of the work I have done in this program has been focused on
implementing commonly used techniques, such as frameworks for
estimating and optimizing equations. Most of the focus of the work I
have done has been in convex programming because of its ease of
solving and global optimality from first-order conditions. I have
worked also in integer programming and some more complicated nonlinear
functions. However, the focus was not on those programs due to their
inherit complexity and inability to solve at scale. I believe that I
am quite qualified in numerical methods and computation, but would
like to continue exploring their application to economic models.

During the course of this year, one aspect that has piqued my interest
has been mechanism design. I believe that many of the strange
behaviors that we observe today are the result of misaligned
incentives and that mechanism design is the way we can control, or at
least contextualize, these behaviors. My experience of it has been
limited, primarily in auction theory and in the context of insurance
models during information economics. I would certainly like to study it
more and explore its results and limitations outside the scope of
these well-understood models.

However, there is still a lot of economics I want to learn before I
decide what it is I want to do. I would be perfectly content working
in Econometrics, as I feel I would excel at it, but I would like to
take classes in other branches as well before I decide what I do. This
is why I would like to enroll in the masters to improve upon my
knowledge of Micro and develop some experience in
Macroeconomics. 

I would like to continue graduate school after this program, and I
believe that the Masters degree in econometrics is the best degree to do
this from. I am prepared for a mathematically, and computationally
rigorous program, and I believe I would excel in a program such as
that. However I understand that I may not be prepared for a program as
rigorous as this, and I may need to become more familiar with
difficult programs. This is why I am also applying for the masters in
Economics, which I understand is not an easy program, but I may find
less strenuous than the masters in econometrics and mathematical
economics. I believe I would be challenged in either program, and
would rise to that challenge to succeed no matter where I am.
% Emacs 25.3.1 (Org mode 8.2.10)
\end{document}