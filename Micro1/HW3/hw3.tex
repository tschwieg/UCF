\documentclass[10pt, letterpaper]{paper}
\usepackage[margin=0.5in]{geometry}
\usepackage{graphicx}
\usepackage{amsmath}
\usepackage{amssymb}
\usepackage{mathrsfs}
\usepackage{blindtext}
\usepackage[utf8]{inputenc}
\usepackage{forest}
\usepackage{float}

\title{ Micro Theory HW3 }
\author{ Timothy Schwieg }
\date{ November 5th 2017 }


\DeclareMathOperator*{\maxi}{argmax}

\begin{document}

\maketitle


\section*{Question 1}
\subsection*{a}
The profit for firm j is given by: $\pi^j = pq - (aq + bq^2)$
\newline
The firm's supply function is given by:
 $q^j= \maxi_{q} \quad \pi^j ( q; p,a,b) \quad s.t. \quad q \geq 0$
\newline
$q^j= \maxi_{q} \quad (p-a)q - bq^2 \quad s.t \quad q \geq 0$  this does not have a maximum for $b < 0$. This is an upwards facing parabola, so firms can always produce more goods to have a higher profit, regardless of the price in the market.
\newline
Let $b > 0$
$q^j = \max\{ 0, \frac{ p - a }{2b} \}$
\newline
The Industry's Supply function is given by: $\sum_{j \in J} \max\{ 0, \frac{ p-a}{2b} \} = \max\{ 0, J \frac{ p-a}{2b}\}$
\newline
Setting Supply equal to Demand:
$max\{ 0, J \frac{ p-a}{2b}\} = 1 - p$
\newline
Case: $p > a: \quad Jp - aJ = 2b - 2bp$
\newline
$p( J + 2b) = 2b + aJ \implies p = \frac{ 2b + aJ }{J + 2b}$
\newline
Note that this is consistent with $p > a$ since: $ a = \frac{ aJ + 2ba }{J + 2b} < \frac{2b + aJ}{J + 2b} = p$
\newline
Case $p \leq a: 0 = 1-p \implies p = 1 > a \quad$This is inconsistent with $p \leq a$ so: $p > a$ always.
\newline
$q = 1- p = \frac{ J + 2b }{ J + 2b} - \frac{ 2b + aJ }{J + 2b} = \frac{ J(1-a)}{J + 2b}$
\newline

tiable, and however, from plotting indifference curves, one can see that the maximum will occur at: $2x^1 = y^1$
\newline
Since: $p_1 x^1 + p_2 y^1 = p_1 A, p_1 x^1 + 2 p_2 x^1 = p_1 A$
\newline
$x^1 = \frac{ p_1 A }{ p_1 + 2p_2 }, y^1 = \frac{ 2p_1 A}{p_1 + 2p_2}$ 
\newline
Consumer 2 faces a similiar problem and will see his maximum at: $x^2 = 2 y^2$.
\newline
His demand functions are: $ x^2 = \frac{ 2p_2 B }{ 2p_1 + p_2}, y^2 = \frac{ p_2 B }{ 2p_1 + p_2 }$
\newline \newline
A Walrasian Equilibrium is characterized by excess demand being zero at the prices, so we construct the excess demand vector and set it equal to 0.
\newline
$Z_1 = \frac{ p_1 A }{ p_1 + 2p_2} + \frac{ 2p_2 B}{ 2p_1 + p_2 } - A = 0$
\newline
$Z_2 = \frac{ 2 p_1 A }{ p_1 + 2p_2 } + \frac{ p_2 B }{ 2p_1 + p_2 } - B = 0$
\newline
Expanding $Z_1: 2p_1^2 A + p_1 p_2 A + 2p_2 p_1 B + 4p_2^2 B = 2Ap_1^2 + 5Ap_1 p_2 + 2Ap_2^2$
\newline
Factoring into quadratic form:  
\[
	p^T
	\left[ {\begin{array}{cc}
	2A & \frac{ A + 2B }{2} \\
	\frac{ A + 2B}{2} & 4B \\
	\end{array} } \right]
	p = p^T
	\left[ {\begin{array}{cc}
	2A & \frac{5A}{2} \\
	\frac{5A}{2} & 2A \\
	\end{array} } \right] p \quad \implies
	p^T
	\left[ {\begin{array}{cc}
	0 & B - 2A \\
	B - 2A & 4B - 2A \\
	\end{array} } \right]
	p = 0
\]
\newline
This has solutions of the form: $p_1 = \frac{ p_2 (2B - A) }{2A - B}$
\newline
We are seeking positive prices: So either $2B - A > 0$ and $2A - B > 0$, or we see: $2B - A < 0$ and $2A - B < 0$
\newline
However, If we consider the negative option: $4B < 2A, 2A < B$. This implies: $4B < B$ which is inconsistent with positive endowments. Thus the only criterion is:
$2B - A > 0$ and $2A - B > 0$.
\newline
The equilibrium allocations for each consumer are:

\begin{equation*}
\begin{alignedat}{8}
x^1 &= \frac{ p_1 A }{ p_1 + 2p_2 } &= \frac{ p_2 A ( 2B - A) (2A-B)}{(2A -B)(p_2 (2B-A) + 2p_2 (2A - B) )} \quad &= \frac{ A( 2B - A)}{ 2B - A + 4A - 2B} \quad &= \frac{ 2B - A }{3}\\
y^2 &= \frac{ p_2 B }{ 2p_1 + p_2 } &=\frac{ p_2 B (2A- B)}{ 2p_2 (2B - A) + p_2 (2A - B) } &= \frac{ B( 2A - B)}{4B - 2A + 2A - B} &=\frac{ 2A -B}{3}\\
y^1 &= \frac{ 4B - 2A }{3} . \quad &x^2 = \frac{ 4A - 2B}{3}  & & \\
\end{alignedat}
\end{equation*}


\subsection*{b}
$V^1( x^1, y^1 ) = min\{ 2x^1, y^1 \} = \frac{ 4B - 2A }{3}$ 
\newline
Note that this function is now differentiable, so we may take comparative statics:
\newline
$\frac{ \partial V^1 }{ \partial A } = \frac{ -2 }{3} < 0$
Thus consumer 1's utility is a decreasing function of A, and reducing A would lead to an increased level of utility.
Intuitively, Having less A reduces the price of B by so much that consumer 1 actually obtains more $x^1$ and $y^1$ depsite losing endowment.

\section*{Question 4}
\subsection*{a}
The set of Pareto Optima for this economy is the singleton allocation x where cosumer 1 has: $(5,0)$ and Consumer 2 has: $(0,7)$. First we will show that this allocation is Pareto Efficient, then we will demonstrate that it is unique.
Consider any other feasible allocation, since it is feasible, the allocations must sum to $(5,7)$
So this allocation must either have less of good 1 for consumer 1 or less of good 2 for consumer 2.
Without loss of generality, assume consumer 1 has less than five units of good 1. Since consumer 2 is
not any worse off by giving up his good 1, and consumer 1 is strictly better off, Thus x is Pareto Efficient.
\newline
We will establish uniqueness via Contradiction. Suppose there is another Pareto efficient allocation $y \neq x$.
Since y is feasible, and $y \neq x$, By the previous logic, bundle x has one consumer better off, with no consumer worse off.
Thus y cannot be a Pareto efficient allocation, and x must be unique.
\subsection*{b}
Let $U_1 (x_1,x_2 ) = x_1, U_2 ( y_1, y_2 ) = y_2 \quad e_1 = (1,2)^T \quad e_2 = (4,5)^T$
\newline
Consumer 1 faces his decision problem of: $max x_1 s.t. \quad p_1 x_1 + p_2 x_2 = p_1+ 2p_2, x_1 \geq 0, x_2 \geq 0$
\newline
Solving the constraint for $x_1: x_1 = \frac{ p_1 + p_2 ( 2- x_2) }{p_1}$ This is maximized when $x_2 = 0.$ Thus: $x_1 = \frac{ p_1 + 2p_2}{p_1}$.
\newline \newline
Consumer 2 faces a similiar problem of: $max y_2 s.t. \quad p_1 x_1 + p_2 x_2 = 4_1 + 5p_2$
\newline
Solving for $y_2: y_2 = \frac{ 5p_2 + p_1 ( 4 - y_1 ) }{p_2}$. This is maximized when $y_1 = 0$. $y_2 = \frac{ 5p_2 +4p_1 }{p_2}$.
\newline \newline
Setting Excess demand equal to 0 for good 1:
\begin{align*}
\frac{ p_1 + 2p_2 }{p_1} &= 5\\
p_1 + 2p_2 &= 5p_1\\
p_2 &= 2 p_1\\
x_1 &= \frac{p_1 + 4p_1 }{p_1} = 5\\
x_2 &= 0\\
y_1 &= 0\\
y_2 &= \frac{ 10p_1 + 4p_1 }{2p_1} = 7\\
\end{align*}




\end{document}