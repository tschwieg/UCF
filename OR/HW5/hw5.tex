\documentclass[10pt, letterpaper]{paper}
\usepackage[margin=0.5in]{geometry}
\usepackage{graphicx}
\usepackage{amsmath}
\usepackage{amssymb}
\usepackage{mathrsfs}
\usepackage{blindtext}
\usepackage[utf8]{inputenc}
\usepackage{forest}
\usepackage{float}
\input julialisting.tex

\lstset{
  columns=fullflexible,
  breaklines=true,
  postbreak=\mbox{\textcolor{red}{$\hookrightarrow$}\space},
}


\title{ Operations Research HW5 }
\author{ Timothy Schwieg }
\date{ November 5th 2017 }

\begin{document}

\maketitle

\section*{Question 1 }

If we let the cost vector be denoted c, the time spent vector be denoted P, due date denoted D, we may construct artifical variables $s_i$ which is the delay on job i, and $y_{ij}$ is an indicator if job i precedes job j, as well as Z which is 1 if job 4 precedes job 3.
\newline
This problem becomes:
\begin{equation*}
\begin{alignedat}{3}
&\text{min }&c^T s&\\
&\text{s.t. } &x_i - s_i&\leq D_i - P_i\\
& &M y_{ji} + x_j - x_i  &\geq P_i\\
& &M y_{ij} + x_i - x_j  &\geq P_j\\
& &y_{ij} + y_{ji} &= 1\\
& &x_3 - x_4 - P_4 &\leq M( 1 - Z ) - \epsilon\\
& &x_9 + P_9 - x_3 &\leq Z M\\
& &x,s \in \mathbb{R}_+ \quad& y_{ij},Z \in \{0,1\}\\
\end{alignedat}
\end{equation*}
Since there are so many constraints, Julia code to solve this will be generated in python.

\lstinputlisting[language=python]{printJulia.py}

Solving this system using the computer generated code in Julia:

%\lstinputlisting[language=Julia]{juliacode.txt}


\begin{lstlisting}[language=sh]
julia> println( "Objective Value: ", getobjectivevalue(m))
Objective Value: 318.0

julia> println( getvalue(x))
x: 1 dimensions:
[0] = 0.0
[1] = 82.99999999999999
[2] = 86.0
[3] = 115.99999999999999
[4] = 32.0
[5] = 10.0
[6] = 99.0
[7] = 41.0
[8] = 71.0
[9] = 130.99999999999997


julia> println( getvalue(s))
s: 1 dimensions:
[0] = 0.0
[1] = 0.0
[2] = 0.0
[3] = 96.99999999999999
[4] = 0.0
[5] = 0.0
[6] = 84.0
[7] = 11.0
[8] = 2.9999999999999925
[9] = 0.0
\end{lstlisting}





\section*{Question 2}

We set $x_1,x_2,x_3$ to be production of product 1,2,3 respectively, and let $s_1$ be a binary predictor of if product 3 is produced.
It is clear that $x_3 \leq 100$ so by setting $x_3 \leq 100 s_1$ this ensures $x_3$ = 0 if $s_1$ = 0, and $x_3$ is otherwise unaffected.

\begin{equation*}
\begin{alignedat}{3}
&\text{max }&25x_1 + 30x_2 + 45x_3&\\
&\text{s.t. } &3x_1 + 4x_2 + 5x_3  &\leq 100\\
& &4x_1+3x_2+6x_3  &\leq 100\\
& &x_3 - 5s_1 &\geq 0\\
& &x_3 - 100s_1 &\leq 0\\
& &x_1, x_2, x_3& \in \mathbb{R}_+, s_1 \in \{0,1\}\\
\end{alignedat}
\end{equation*}

Solving the Relaxed Linear Program:

\begin{equation*}
\begin{alignedat}{3}
&\text{max }&25x_1 + 30x_2 + 45x_3&\\
&\text{s.t. } &3x_1 + 4x_2 + 5x_3  &\leq 100\\
& &4x_1+3x_2+6x_3  &\leq 100\\
& &x_3 - 5s_1 &\geq 0\\
& &x_3 - 100s_1 &\leq 0\\
& &s_1 &\leq 1\\
& &x_1, x_2, x_3, s_1& \in \mathbb{R}_+\\
\end{alignedat}
\end{equation*}

This has a maximal value of: $\frac{2500}{3}$ at a maximizer of: $x^* = (0,\frac{100}{9},\frac{100}{9}), s_1 = \frac{1}{9}$


Branching on $s_1 = 0$ and $s_1 = 1$
\newline
Case: $s_1 = 1$
\begin{equation*}
\begin{alignedat}{3}
&\text{max }&25x_1 + 30x_2 + 45x_3&\\
&\text{s.t. } &3x_1 + 4x_2 + 5x_3  &\leq 100\\
& &4x_1+3x_2+6x_3  &\leq 100\\
& &x_3 - 5s_1 &\geq 0\\
& &x_3 - 100s_1 &\leq 0\\
& &s_1 &\leq 1\\
& &s_1 &\geq 1\\
& &x_1, x_2, x_3, s_1& \in \mathbb{R}_+\\
\end{alignedat}
\end{equation*}

This has a maximal value of: $\frac{2500}{3}$ at a maximizer of: $x^* = (0,\frac{100}{9},\frac{100}{9}), s_1 = 1$
\newline
\newline 
Case: $s_1 = 0$
\begin{equation*}
\begin{alignedat}{3}
&\text{max }&25x_1 + 30x_2 + 45x_3&\\
&\text{s.t. } &3x_1 + 4x_2 + 5x_3  &\leq 100\\
& &4x_1+3x_2+6x_3  &\leq 100\\
& &x_3 - 5s_1 &\geq 0\\
& &x_3 - 100s_1 &\leq 0\\
& &s_1 &\leq 0\\
& &x_1, x_2, x_3, s_1& \in \mathbb{R}_+\\
\end{alignedat}
\end{equation*}

This has maximal value of: $\frac{5500}{7}$ at a maximizer of: $x^* = (\frac{100}{7},\frac{100}{7},0), s_1 = 0$
\newline
\newline
Since $\frac{2500}{3} > \frac{ 5500}{7}$ We choose to utilize $x_3$ and produce: $\frac{2500}{3} $ at a maximizer of: $x^* = (0,\frac{100}{9},\frac{100}{9})$

The Tree for this process is shown below:
\newline
\begin{forest}
  [Start[{$s_1$ = 0}[{$(\frac{100}{7},\frac{100}{7},0)$}[{Z=$\frac{5500}{7}$}]]][{$s_1$ = 1}[{$(0,\frac{100}{9},\frac{100}{9})$}[{Z=$\frac{2500}{3}$}]]]]
\end{forest}

\section*{Question 3}
Solving the Linear Program with no Integer constraints. 
We arrive at the optimal solution of $Z=\frac{311}{7}$ where the maximizer is $x=(1,\frac{3}{7},0,0,0,1)$
\newline
Branching on $x_2 = 0,1$
\newline
\indent Case: $x_2 = 0:$ We reach the optimal solution of: 44 located at: $x=(1,0,\frac{3}{4},0,0,1)$
\indent \newline
\indent Branching upon $x_3 = 0,1$
\indent \newline
\indent \indent Case: $x_3 = 0:$ We arrive at: $Z = 43.25$ located at: $x=(1,0,0,0,\frac{3}{4},1)$
\indent \indent \newline
\indent \indent Branching upon $x_5 = 0,1$
\indent \indent \newline
\indent \indent \indent Case: $x_5 = 0.$ Z=43 located at: $x=(1,0,0,1,0,1)$ This is our Best Feasible Solution.
\indent \indent \indent \newline
\indent \indent \indent Case: $x_5 = 1. Z=42.8333$ This is below our current best Feasible Solution: 43.
\indent \indent \indent \newline
\indent \indent Case: $x_3 = 1: Z = \frac{263}{6}$ located at: $x=(1,0,1,0,0,\frac{5}{6})$
\indent \indent \newline
\indent \indent Branching upon: $x_6 = 0,1$
\indent \indent \newline
\indent \indent \indent Case: $x_6 = 0. Z=41.6667$ This is below our current best Feasible Solution: 43
\indent \indent \indent \newline
\indent \indent \indent Case: $x_6 = 1. Z=43.8$ located at: $x=( \frac{4}{5},0,1,0,0,1)$
\indent \indent \indent \newline
\indent \indent \indent Branching on $x_1 = 0,1$
\indent \indent \indent \newline
\indent \indent \indent \indent Case: $x_1 = 0. Z=42.$ This is below BFS.
\indent \indent \indent \indent \newline
\indent \indent \indent \indent Case: $x_1 = 1.$ This is infeasible.
\indent \indent \indent \indent \newline
\indent Case: $x_2 = 1:$ This has solution: 44.333 located at: $x= (1,1,0,0,0,\frac{1}{3})$
\indent \newline
\indent Branching upon $x_6 = 0,1$
\indent \newline
\indent \indent Case: $x_6 = 0: Z=44$ located at $x=(1,1,\frac{1}{2},0,0,0)$
\indent \indent \newline
\indent \indent Branching on $x_3 = 0,1$
\indent \indent \newline
\indent \indent \indent Case: $x_3 = 0: Z=43.5$ located at $x=(1,1,0,0,\frac{1}{2},0$
\indent \indent \indent \newline
\indent \indent \indent Branching on $x_5 = 0,1$
\indent \indent \indent \newline
\indent \indent \indent \indent Case: $x_5 = 0.$ $Z = 43.333$ located at $x=(1,1,0,\frac{2}{3},0,0)$
\indent \indent \indent \indent \newline
\indent \indent \indent \indent Branching on $x_4 = 0,1$
\indent \indent \indent \indent \newline
\indent \indent \indent \indent \indent Case: $x_4 = 0. Z = 38.$ This is below our BFS
\indent \indent \indent \indent \indent \newline
\indent \indent \indent \indent \indent Case: $x_4 = 1. Z = 42.8.$ This is below our BFS
\indent \indent \indent \indent \indent \newline
\indent \indent \indent \indent Case: $x_5 = 1. Z = 42.6.$ This is below our BFS
\indent \indent \indent \indent \newline
\indent \indent \indent Case: $x_3 = 1. Z=43.6$ located at: $x=(\frac{3}{5},1,1,0,0,0)$
\indent \indent \indent \newline
\indent \indent \indent Branching upon $x_1 = 0,1$
\indent \indent \indent \newline
\indent \indent \indent \indent Case: $x_1 = 0. Z=42.25.$ This is below our BFS
\indent \indent \indent \indent \newline
\indent \indent \indent \indent Case: $x_1 = 1.$ This is Infeasible.
\indent \indent \indent \indent \newline
\indent \indent Case: $x_6 = 1. Z=44.2$ located at $x=(\frac{1}{5},1,0,0,0,1)$
\indent \indent \newline
\indent \indent Branching on $x_1 = 0,1$
\indent \indent \newline
\indent \indent \indent Case: $x_1 = 0. Z=44$ at $x=(0,1,\frac{1}{4},0,0,1)$
\indent \indent \indent \newline
\indent \indent \indent Branching on $x_3 = 0,1$
\indent \indent \indent \newline
\indent \indent \indent \indent Case: $x_3 = 0. Z=43.75$ at $x=(0,1,0,0,\frac{1}{4}, 1)$
\indent \indent \indent \indent \newline
\indent \indent \indent \indent Branching on $x_5 = 0,1$
\indent \indent \indent \indent \newline
\indent \indent \indent \indent \indent Case: $x_5 = 0. Z = 43.667$ at $x=(0,1,0,\frac{1}{3},0,1)$
\indent \indent \indent \indent \indent \newline
\indent \indent \indent \indent \indent Branching on $x_4 = 0,1$
\indent \indent \indent \indent \indent \newline
\indent \indent \indent \indent \indent \indent Case: $x_4 = 0. Z = 41.$ This is below our BFS
\indent \indent \indent \indent \indent \indent \newline
\indent \indent \indent \indent \indent \indent Case: $x_4 = 1.$ Infeasible
\indent \indent \indent \indent \indent \indent \newline
\indent \indent \indent \indent \indent Case: $x_5 = 1$ Infeasible
\indent \indent \indent \indent \indent \newline
\indent \indent \indent \indent Case: $x_3 = 1$ Infeasible
\indent \indent \indent \indent \newline
\indent \indent \indent Case: $x_1 = 1. $ Infeasible
\newline \newline 
Our best Feasible Solution is Z=43 located at: $x=(1,0,0,1,0,1)$


\section*{Question 4}
Let $b_0 = a_0$ mod d, $b_j = a_j$ mod d
\subsection*{a}
$31x_1 + 32x_2 + 9x_3 - 29x_4 = 51$, d = 15
\newline
$x_1 + 2x_2 + 9x_3 + x_4 \geq 6$
\subsection*{b}
$20x_1 + 30x_2 - 5x_3 + 17x_4 + 11x_5 = 168$, d = 17
\newline
$3x_1 + 13x_2 + 12x_3 + 11x_5 \geq 15$




\section*{Question 5}

\begin{equation*}
\begin{alignedat}{3}
&\text{max }&3x_1 + 2x_2&\\
&\text{s.t. } &4x_1 +2x_2&\leq 15\\
& &x_1 + 2x_2  &\leq 8\\
& &x_1 + x_2 &\leq 5\\
& &x_1,x_2 \in \mathbb{Z}_+ &\\
\end{alignedat}
\end{equation*}

This has a relaxed linear program of:
\begin{equation*}
\begin{alignedat}{3}
&\text{max }&3x_1 + 2x_2&\\
&\text{s.t. } &4x_1 +2x_2 + s_1&= 15\\
& &x_1 + 2x_2 + s_2  &= 8\\
& &x_1 + x_2 + s_3 &= 5\\
& &x_1,x_2,s_1,s_2,s_3 \in \mathbb{R}_+ &\\
\end{alignedat}
\end{equation*}



Putting it into Clean Table form
\[
	\left[ {\begin{array}{cccccc|c}
	Z & x_1 & x_2 & s_1 & s_2 & s_3 &  RHS\\ \cline{1-7}
	1 & -3 & -2 & 0 & 0 & 0  &0 \\
	0 & 4 & 2 & 1 & 0 & 0 & 15 \\
	0 & 1 & 2 & 0 & 1 & 0 & 8 \\
	0 & 1 & 1 & 0 & 0 & 1 & 5  \\
	\end{array} } \right]
\]

\[
	\left[ {\begin{array}{cccccc|c}
	Z & x_1 & x_2 & s_1 & s_2 & s_3 &  RHS\\ \cline{1-7}
	1 & 0  & \frac{-1}{2} & \frac{3}{4} & 0 & 0  &\frac{45}{4} \\
	0 & 1 & \frac{1}{2} & \frac{1}{4} & 0 & 0 & \frac{15}{4} \\
	0 & 0 & \frac{3}{2} & \frac{-1}{4} & 1 & 0 & \frac{17}{4} \\
	0 & 0 & \frac{1}{2} & \frac{-1}{4} & 0 & 1 & \frac{5}{4}  \\
	\end{array} } \right]
\]

\[
	\left[ {\begin{array}{cccccc|c}
	Z & x_1 & x_2 & s_1 & s_2 & s_3 &  RHS\\ \cline{1-7}
	1 & 0 & \frac{-1}{2} & \frac{3}{4} & 0 & 0  &\frac{45}{4} \\
	0 & 1 & \frac{1}{2} & \frac{1}{4} & 0 & 0 & \frac{15}{4} \\
	0 & 0 & \frac{3}{2} & \frac{-1}{4} & 1 & 0 & \frac{17}{4} \\
	0 & 0 & \frac{1}{2} & \frac{-1}{4} & 0 & 1 & \frac{5}{4}  \\
	\end{array} } \right]
\]

\[
	\left[ {\begin{array}{cccccc|c}
	Z & x_1 & x_2 & s_1 & s_2 & s_3 &  RHS\\ \cline{1-7}
	1 & 0 & 0 & \frac{1}{2} & 0 & 0  &\frac{25}{2} \\
	0 & 1 & 0 & \frac{1}{2} & 0 & -1 & \frac{5}{2} \\
	0 & 0 & 0 & \frac{1}{2} & 1 & -3 & \frac{1}{2} \\
	0 & 0 & 1 & \frac{-1}{2} & 0 & 2 & \frac{5}{2}  \\
	\end{array} } \right]
\]

This is the final tableau for the relaxed linear program. Since it yeilds maximizers of:
$x_1 = \frac{5}{2}, x_2 = \frac{5}{2},$ we are not at a feasible integer program solution, and we will take the first constraint from the tableau to make our Gomory cut.
\newline
$x_1 + \frac{1}{2} s_1 - s_3 = \frac{5}{2}$
\newline
$(1+\frac{0}{1})x_1 + (0 + \frac{1}{2})s_1 + (-1 + \frac{0}{1})s_3 = (2 + \frac{1}{2})$
\newline
So by adding the constraint: $\frac{1}{2} s_1 \geq \frac{1}{2}$ We have a new LP.

\begin{equation*}
\begin{alignedat}{3}
&\text{max }&3x_1 + 2x_2&\\
&\text{s.t. } &4x_1 +2x_2 + s_1&= 15\\
& &x_1 + 2x_2 + s_2  &= 8\\
& &x_1 + x_2 + s_3 &= 5\\
& &\frac{1}{2} s_1 &\geq \frac{1}{2}\\
& &x_1,x_2,s_1,s_2,s_3 \in \mathbb{R}_+ &\\
\end{alignedat}
\end{equation*}
This has solution Z = 12 located at $x = (2,3,1,0,0)$.

\section*{Question 6}
\subsection*{a}
$\frac{5}{2} y_1 + \frac{7}{3} y_2 + \frac{1}{5} y_3 + 2x_1 \leq \frac{17}{4} + x_2$
\newline
We can see that: $f = \frac{1}{4}, f_1 = \frac{1}{2}, f_2 = \frac{1}{3}, f_3 = {1}{5}$
\newline
So our MIR is: $(2+\frac{\frac{1}{4}^+}{\frac{3}{4}})y_1 + (2+\frac{\frac{1}{12}^+}{\frac{3}{4}})y_2 + (0 + \frac{(\frac{1}{5} -\frac{1}{4})^+}{\frac{3}{4}})y_3 \leq 4 + \frac{x_2}{\frac{3}{4}}$
\newline
This simplifies to: $\frac{7}{3} y_1 + \frac{19}{9} y_2 \leq 4 + \frac{4x_2}{3}$

\subsection*{b}
$\frac{5}{2}y_1 + \frac{5}{2} y_2 + \frac{19}{6}y_3 - x_1 \leq \frac{73}{5} + 2x_2$
\newline
$\frac{5}{2}y_1 + \frac{5}{2} y_2 + \frac{19}{6}y_3 \leq \frac{73}{5} + x_1 + 2x_2 $
\newline
We can see that: $f = \frac{3}{5}, f_1 = \frac{2}{3}, f_2 = \frac{1}{2}, f_3 = {1}{6}$
\newline
So our MIR is: $(2+\frac{(\frac{2}{3} - \frac{3}{5})^+}{\frac{2}{5}})y_1 + (2+\frac{(\frac{1}{2}-\frac{3}{5})^+}{\frac{2}{5}})y_2 + (3 + \frac{(\frac{1}{6} - \frac{3}{5})^+}{\frac{2}{5}})y_3 \leq 14 + \frac{x_1 + 2x_2}{\frac{2}{5}}$
\newline
This simplifies to: $\frac{13}{6} y_1 + 2y_2 + 3y_3 \leq 14 + \frac{5}{2}x_1 + 5x_2$
\end{document}